% !TeX encoding = UTF-8
% chktex-file 21 chktex-file -2 chktex-file 23
\documentclass[fontset=ubuntu,12pt,a4paper]{scrreprt}
\usepackage[UTF8]{ctex}
\usepackage[utf8]{inputenc}
\usepackage{ntheorem}
\usepackage[ngerman]{babel}
\usepackage{amsmath}
\usepackage{amsfonts}
\usepackage{amssymb}
\usepackage{enumitem}
\usepackage{commath}
\usepackage{tikz}
\usepackage{pgfplots}
\usepackage[autostyle=true,german=quotes]{csquotes}
\usepackage[pdftex,pdfa,hidelinks]{hyperref}
\newtheorem{defi}{Definition}[section]
\newtheorem{bemerkung}[defi]{Bemerkung}
\newtheorem{beispiel}[defi]{Beispiel}
\newtheorem{satz}[defi]{Satz}
\newtheorem{aufg}[defi]{Aufgabe}
\newtheorem*{einschub}{Einschub}

\pgfplotsset{compat=1.16}


\begin{document}
    
    \tableofcontents

    \chapter{Grundlagen der Differentialrechnung reeller Funktionen mit mehreren Variablen}
    
    \section{Partielle Ableitungen}
    
    \begin{defi}[reelle Funktionen mit n Variablen]
        Eine Funktion 
        
        \(y = f(x_1, \dots, x_n)\) mit \(\left(x_1, \dots, x_n\right) \mathbb{D} \subset \mathbb{R}^n\) und \(y \in \mathbb{R}\) heißt reelle Funktion mit mehreren Variablen.

        \(\mathbb{D}\) beschreibt den Definitionsbereich und wir schreiben \(f:\mathbb{D} \to \mathbb{R}\)
    \end{defi}

\begin{bemerkung}
    Definition für Differenzierbarkeit im eindimensionalen Fall:
    Es sei \(f : \mathbb{D} \to \mathbb{R}\) eine Funktion mit \(\mathbb{D} \subset \mathbb{R}\) und \(x_0 \in \mathbb{R}\)
    f ist differenzierbar in \(x_0\), falls der Grenzwert \(\lim_{x\to x_0} {\frac{f\left(x\right) - f\left(x_0\right)}{x-x_0}} \in \mathbb{R}\) existiert.
    In diesem Fall heißt \(f'\left(x_0\right) = \lim_{x\to x_0}{\frac{f\left(x\right) - f\left(x_0\right)}{x-x_0}}\) die Ableitung von f in \(x_0\)
    Man bezeichnet \(f'\left(x_0\right)\) als den Differentialquotienten von f im Punkt \(x_0\).
    
    \(f'\left(x_0\right) = \lim_{\Delta x\to0}{\frac{f\left(x_0 + \Delta x\right) - f\left(x_0\right)}{\Delta x}}\)
    
\end{bemerkung}

\begin{defi}
    \(P=(p_1,\ldots,p_n)\) und \(Q=(q_1,\ldots,q_n)\) bezeichnen zwei Punkte im n-dimensionalen Raum \(\mathbb{R}^n\)
    
    \(\left|P-Q\right|=\sqrt{{\left(p_1-q_1\right)}^2 + \cdots + {\left(p_n+q_n\right)}^2}\) heißt Abstand der Punkte P und Q.
    
    Die Delta-Umgebung des Punktes P ist eine Teilmenge des \(\mathbb{R}^n\) mit der Eigenschaft \(U_\delta\left(P\right)=\left\{Q\in\mathbb{R}^n:\left|Q-P\right|<\delta\right\}\).
    
\end{defi}
    \begin{defi}
        Der Punkt P heißt innerer Punkt der Menge M \((M \subset R^n)\), wenn eine Umgebung des Punktes P existiert, für die \(U_\delta \subset M\) gilt.
    \end{defi}

\begin{bemerkung}
    Eine Menge heißt \underline{offene Menge}, wenn dir nur aus inneren Punkten besteht.
\end{bemerkung}

\begin{defi}
    Wenn \((x_{0n},\dots,x_{0n})\) ein innerer Punkt der Menge D ist und wenn der Grenzwert existiert, dann heißt die Funktion \(f(x_1,\dots,x_n)\) an der Stelle \((x_{01},\dots,x_{0n})\in D\) nach \(x_i\) partiell differenzierbar. Den Grenzwert bezeichnet man als partielle Ableitung der Funktion f nach \(x_i\) an der Stelle \((x_{01},\dots,x_{0n}) \in D\)
    
    Die Funktion f heißt in \((x_{01},\dots,x_{0n}) \in D\) partiell differenzierbar, wenn die partiellen Ableitungen nach allen Komponenten \(x_j (j=1,\dots,n)\) existieren.
    
    Die Funktion heißt in \(\mathbb{D}\) partiell differenzierbar, wenn \(f\) in allen inneren Punkten aus \(\mathbb{D}\) partiell differenzierbar ist.
\end{defi}

\begin{bemerkung}
    Die partielle Ableitung der Funktion \(f(x_1,\dots,x_n)\) nach der Komponente \(x_j\) kann wie folgt bezeichnet werden:
    
    \(f_{x_j}(x_1,\dots,x_n)\) oder \(\pd{f(x_1,\dots,x_n)}{x_j}\)
\end{bemerkung}
\begin{beispiel}
    Gesucht sind die partiellen Ableitungen der Funktion
    \(f(x_1,x_2,x_3)=x_3*\sin(x_1^2+x_2)+e^{2x_3}\)
    
    \[\dpd{f}{x_1}(x_1,x_2,x_3)=x_3\cdot\cos(x_1^2+x_2)\cdot2x_1\]
    
    \[\dpd{f}{x_2}(x_1,x_2,x_3)=x_3\cdot\cos(x_1^2+x_2)\]
        
    \[\dpd{f}{x_3}(x_1,x_2,x_3)=\sin(x_1^2+x_2)\cdot2e^{2x_3}\]
\end{beispiel}

\begin{bemerkung}
    Die Tangentialebene an die Funktion \(f(x_1,x_2)\) berührt die Funktion \(f(x_1,x_2)\) im Punkt \(\bar{P}=(\bar{x_1},\bar{x_2},f(\bar{x_1},\bar{x_2}))\) und enthält alle Tangenten an die Funktion \(f(x_1,x_2)\) im Punkt \(\bar{P}\).
\end{bemerkung}

\begin{satz}
Es seien \(\mathbb{D} \subset \mathbb{R}^2, f:\mathbb{D}->\mathbb{R}\) eine partiell differenzierbare Funktion und \((x_{01},x_{02})\in\mathbb{D}\). Dann lautet die Gleichung der Tangentialebene für den Punkt \((x_{01},x_{02})\):

\[x_3=f(x_{01},x_{02})+\dpd{f}{x_1}(x_{01},x_{02})(x_1-x_{01})+\dpd{f}{x_2}(x_{01},x_{02})(x_2-x_{02})\]

\[=f(x_{01},x_{02})+\begin{pmatrix} \dpd{f}{x_1} (x_{01},x_{02}) & \dpd{f}{x_2}(x_{01},x_{02}) \end{pmatrix} \cdot \begin{pmatrix} (x_1-x_{01})\\ (x_2-x_{02}) \end{pmatrix}\]
\end{satz}

\begin{beispiel}
    Gegeben sei \(f:\mathbb{R}^2\to\mathbb{R}\) durch \(f(x_1,x_2)=\sin(x_1\cdot x_2^2)\)

    \(\pd{f}{x_1}(x_1,x_2)=\cos(x_1\cdot x_2^2)\cdot x_2^2\)
    
    \(\pd{f}{x_2}(x_1,x_2)=\cos(x_1\cdot x_2^2)\cdot 2x_2 \cdot x_1\)
    
    \(\to\) Tangentialebene: \(x_3 = f(x_{01},x_{02}) + \cos(x_1\cdot x_2^2)\cdot x_2^2 \cdot (x_1 - x_{01}) + \cos(x_1\cdot x_2^2)\cdot 2x_2 \cdot x_1 \cdot (x_2 - x_{02})\)
    
    \((x_{01},x_{02}) = (0,0)\)
    
    \(x_3 = 0 + \begin{pmatrix}0 & 0\end{pmatrix} \begin{pmatrix}x_1 - 0 \\ x_2 - 0\end{pmatrix} = 0\)
    
\end{beispiel}

\begin{beispiel}
    
    \((x_{01},x_{02}) = (\sqrt[3]{\pi},\sqrt[3]{\pi})\)
    
    \(f(\sqrt[3]{\pi},\sqrt[3]{\pi})=\sin(\sqrt[3]{\pi}\cdot \sqrt[3]{\pi}^2) = 0\)
    
    \(\pd{f}{x_1}(\sqrt[3]{\pi},\sqrt[3]{\pi})=\cos(\sqrt[3]{\pi}\cdot \sqrt[3]{\pi}^2)\cdot \sqrt[3]{\pi}^2 = -\pi^{\frac{2}{3}}\)
    
    \(\pd{f}{x_2}(\sqrt[3]{\pi},\sqrt[3]{\pi})=\cos(\sqrt[3]{\pi}\cdot \sqrt[3]{\pi}^2)\cdot 2\sqrt[3]{\pi} \cdot \sqrt[3]{\pi} = -2\pi^{\frac{2}{3}}\)
    
    \(x_3 = 0 + \begin{pmatrix}-\pi^{\frac{2}{3}} & -2\pi^{\frac{2}{3}}\end{pmatrix} \begin{pmatrix}x_1 - \sqrt[3]{\pi} \\ x_2 - \sqrt[3]{\pi}\end{pmatrix} = -\pi^{\frac{2}{3}} \cdot (x_1 - \sqrt[3]{\pi}) + -2\pi^{\frac{2}{3}} \cdot (x_2 - \sqrt[3]{\pi})\)
    
    \(x_3 = -\pi^{\frac{2}{3}}x_1 - 2\pi^{\frac{2}{3}}x_2 + 3\pi\)
\end{beispiel}

\begin{beispiel}
    Gegeben sei \(f:\mathbb{R}^2\to\mathbb{R}\) durch \(f(x_1,x_2)=\left| x_1 \right| + x_2\)
    \(\pd{f}{x_2}(x_1,x_2)=1\)
    \(\left| x_1 \right|\) ist nur für \(x_1 \neq 0\) differenzierbar, d.h. \(f\) ist nicht auf dem gesamten Definitionsbereich partiell differenzierbar.
\end{beispiel}

\begin{defi}[Gradient]
    Es sei \(\mathbb{D} \subset \mathbb{R}^n\) und \(f:\mathbb{D} \to \mathbb{R}\) partiell differenzierbar. Dann heißt der Vektor \(\mathrm{grad} f(x) = \begin{pmatrix}
    \dpd{f}{x_1}(x) \\
    \dpd{f}{x_2}(x) \\
    \vdots \\
    \dpd{f}{x_n}(x) \\
    \end{pmatrix}\) der Gradient von \(f\) im Punkt \(x\in\mathbb{D}= (x_1,\dots,x_n)\).
\end{defi}

\begin{bemerkung}
    Anstelle von \(\mathrm{grad} f(x)\) wird auch häufig \(\nabla f(x)\) geschrieben.
\end{bemerkung}

\begin{beispiel}
    Berechnen Sie den Gradienten für:
    \begin{enumerate}
        \item \(f(x_1,x_2) = x_1^2 + x_2^2\)
        \item \(g(x_1,x_2,x_3) = 2 \cdot \sin(x_1 x_2) + x_1 x_2 x_3\)
    \end{enumerate}

    zu 1. \(\nabla f(x_1,x_2) = \begin{pmatrix}2x_1\\2x_2\end{pmatrix}\)
    
    zu 2. \(\nabla g(x_1,x_2,x_3) = \begin{pmatrix}2x_2 \cos(x_1 x_2)+x_2 x_3 \\ 2x_1 \cos(x_1 x_2)+x_1 x_3 \\ x_1 x_2\end{pmatrix}\)
\end{beispiel}

\begin{beispiel}
    Partielle Differenzierbarkeit impliziert nicht Stetigkeit.
    
    Betrachtet wird die Funktion \(f:\mathbb{R}^2\to\mathbb{R}\) mit
    
    \[f(x_1,x_2) = \begin{cases}
    \frac{x_1\cdot x_2}{x_1^2 + x_2^2} & (x_1,x_2) \ne (0,0) \\
    0 & (x_1,x_2) = (0,0)
    \end{cases}\]
    
    Im Punkt \((0,0)\) existieren die partiellen Ableitungen:
    
    \[\dpd{f}{x_1}(0,0) = \lim_{\Delta x\to0}{\frac{f(0+\Delta x,0)-f(0,0)}{\Delta x}} = 0\]
    
    \[\dpd{f}{x_2}(0,0) = \lim_{\Delta x\to0}{\frac{f(0,0+\Delta x)-f(0,0)}{\Delta x}} = 0\]
    
    Aber \(f\) ist in \((0,0)\) nicht stetig:
    
    Es gilt \(f(x_1,0) = 0; x_1 \in \mathbb{R}\), \(f(0,x_2) = 0; x_2 \in \mathbb{R}\)
    
    Für \(x := x_1 = x_2\):
    
    \(f(x,x)=\begin{cases}\frac{x^2}{2x^2}=\frac{1}{2} & (x,x) \ne (0,0) \\ 0 & (x,x) = (0,0)\end{cases}\)
\end{beispiel}

\begin{defi}[Stetigkeit]
    Die Funktion \(y=f(x_1,\dots,x_n)\), \((x_1,\dots,x_n)\in\mathbb{D}\) ist an der Stelle \(\bar{P}=(\bar{x_1},\dots,\bar{x_n}) \in \mathbb{D}\) stetig, wenn für den Funktionsgrenzwert
    
    \[\lim_{P\to\bar{P}}{f(x_1,\dots,x_n)} = f(\bar{x_1},\dots,\bar{x_n})\] gilt.
\end{defi}

Aufgaben:

\begin{enumerate}
    \item Berechnen Sie alle ersten und zweiten partiellen Ableitungen der Funktion
    
    \(f(x_1,x_2)=e^{-2.5x_1^2-{(x_2-1)}^2}, (x_1,x_2) \in \mathbb{R}\)
    \item Berechnen die den Gradienten der Funktion \(f(x_1,x_2,x_3)=\frac{(x_1-1)\cdot\ln(x_1+1)}{x_2^2+x_3^2+1}\) an der Stelle \((0,0,0)\)
    \item Berechnen Sie die Tangentialebene an die Funktion \\ \(f(x_1,x_2)=e^{-2.5x_1^2-{(x_2-1)}^2}, (x_1,x_2) \in \mathbb{R}\) im Punkt \\ \((0,\frac{3}{2},e^{-\frac{1}{4}})\)
\end{enumerate}

\begin{defi}[k-mal partiell differenzierbar]
    Es sei \(D\subset\mathbb{R}^n,f:D\to\mathbb{R}\) eine partiell differenzierbare Funktion und \(x_0 \in D\). Die Funktion f heißt \\ zweimal \underline{partiell differenzierbar} in \(x_0\), wenn alle partiellen Ableitungen \(\pd{f}{x_i}\) in \(x_0\) wieder partiell differenzierbar sind.

    Man schreibt \(\frac{\partial^2 f}{\partial x_j x_i}(x_0)=\pd{}{x_j}\left(\pd{f}{x_i}\right)(x_0)=f_{x_j x_i}(x_0)\)

    Dieser Ausdruck heißt dann \underline{zweite partielle Ableitung} von f.

    Allgemein heißt f \underline{k-mal partiell differenzierbar}, wennn alle (k-1)-ten partiellen Ableitungen von f wieder partiell differenzierbar sind. Man schreibt:

    \[\dpd{f}{x_{ik} \partial x_{i(k-1)} \dots \partial x_{i1}}(x_0) = \frac{\partial}{\partial x_{ik}}\left(\frac{\partial^{k-1} f}{\partial x_{i(k-1)} \dots \partial x_{i1}}\right)(x_0)=f_{x_{ik} \dots x_{i1}}\]
\end{defi}

\begin{aufg}
    Gegeben sei \(f:\mathbb{R}^2\to\mathbb{R}\) durch \(f(x_1,x_2)=x_1^2 \cdot x_2 - x_1 \cdot x_2^2\)

    Berechnen Sie alle ersten und zweiten Ableitungen von f und \(\frac{\partial^3 f}{\partial x_1 \partial x_1 \partial x_2}(x_1,x_2)\).

    \[\dpd{f}{x_1}(x_1,x_2)=2x_1 \cdot x_2 - x_2^2 \]
    \[\dpd{f}{x_2}(x_1,x_2)=x_1^2 - x_1 \cdot 2x_2 \]

    \[\frac{\partial^2 f}{\partial x_1^2}(x_1,x_2)=2x_2 ; \frac{\partial^2 f}{\partial x_2^2}(x_1,x_2)=-2x_1 \]
    \[\frac{\partial^2 f}{\partial x_1x_2}(x_1,x_2)=2x_1-2x_2 ; \frac{\partial^2 f}{\partial x_2x_1}(x_1,x_2)=2x_1-2x_2 \]
    \[\frac{\partial^3 f}{\partial x_1 \partial x_1 \partial x_2}(x_1,x_2)=+2\]
\end{aufg}

\begin{defi}[stetig partiell differenzierbar]
    Es sei \(D\subset\mathbb{R}^n,f:D\to\mathbb{R}\). \(f\) heißt k-mal \underline{stetig partiell differenzierbar}, falls f k-mal partiell differenzierbar ist und alle partiellen Ableitungen der Ordnung k stetig sind.
\end{defi}

\begin{satz}[Satz von Schwarz]
    Es sei \(D\subset\mathbb{R}^n,f:D\to\mathbb{R}\) zweimal stetig partiell differenzierbar. Dann ist \(\frac{\partial^2 f}{\partial x_i \partial x_j}=\frac{\partial^2 f}{\partial x_j \partial x_i}\) für alle \(i,j\in\left\{1,\dots,n\right\}\). Die Reihenfolge der Ableitungsvariablen spielt also keine Rolle,
\end{satz}

\begin{bemerkung}
    Es sei \(f:D\to\mathbb{R}\) k-mal stetig partiell differenzierbar. Dann spielt die Reihenfolge der Ableitungsvariablen bei der k-ten partiellen Ableitung keine Rolle.
\end{bemerkung}

\begin{aufg}
    Berechnen Sie alle ersten und zweiten partiellen Ableitungen der Funktion \(f(x_1,x_2,x_3)=x_3 \cdot \sin(x_1^2+x_2)+e^{2x_3}\) \((x_1,x_2,x_3)\in\mathbb{R}^3\)

    \(\pd{f}{x_1}(x_1,x_2,x_3) = x_3 \cdot 2x_1 \cdot \cos(x_1^2+x_2)\)

    \(\pd{f}{x_2}(x_1,x_2,x_3) = x_3 \cdot \cos(x_1^2+x_2)\)

    \(\pd{f}{x_3}(x_1,x_2,x_3) = \sin(x_1^2+x_2) + 2e^{2x_3}\)

    \(\frac{\partial^2 f}{\partial x_1^2}(x_1,x_2,x_3) = 2x_3 \left(\cos(x_1^2+x_2) - 2x_1^2\sin(x_1^2+x_2)\right)\)

    \(\frac{\partial^2 f}{\partial x_1 \partial x_2}(x_1,x_2,x_3) = -2x_1x_3 \sin(x_1^2+x_2)\)

    \(\frac{\partial^2 f}{\partial x_2^2}(x_1,x_2,x_3) = -x_3\sin(x_1^2+x_2)\)

    \(\frac{\partial^2 f}{\partial x_1 \partial x_3}(x_1,x_2,x_3) = 2x_1 \cos(x_1^2+x_2)\)

    \(\frac{\partial^2 f}{\partial x_2 \partial x_3}(x_1,x_2,x_3) = \cos(x_1^2+x_2)\)

    \(\frac{\partial^2 f}{\partial x_3^2}(x_1,x_2,x_3) = 4e^{2x_3}\)
\end{aufg}

\section{Totale Differenzierbarkeit}

\begin{defi}[Betrag eines Vektors]
    Der Betrag eines Vektors
    
    \(x=(x_1,\dots,x_n)\in\mathbb{R}^n\) ist definiert als:
    \[\vert x \vert = \sqrt{x_1^2+\cdots+x_n^2}\]
\end{defi}

\begin{defi}[Vektorfunktion]
    Eine eindeutige Abbildung \(f:\mathbb{D}\to W,\mathbb{D}\subset\mathbb{R}^n,W\subset\mathbb{R}^m,m>1\) mit mehrdimensionalem Wertebereich heißt \underline{Vektorfunktion}.
\end{defi}

\begin{beispiel}
    Es sei \(\mathbb{D}\subset\mathbb{R}^3\to\mathbb{R}^2\) gegeben durch
    \[f(x_1,x_2,x_3)=\begin{pmatrix}
    x_1+x_3 \\ x_2+x_3
    \end{pmatrix}\]
    f hat 2 Ergebniskomponenten:
    \[f_1(x_1,x_2,x_3)=x_1+x_2,f_2(x_1,x_2,x_3)=x_2+x_3\]
\end{beispiel}

\begin{defi}[total differenzierbar]
    Es sei \(f:\mathbb{D}\to \mathbb{R}^m,\mathbb{D}\subset\mathbb{R}^n\) eine Abbildung und \(x_0\in\mathbb{D}\). Die Funktion f heißt total differenzierbar in \(x_0\), falls es eine Matrix \(A\in\mathbb{R}^{m\times n}\) und eine Restfunktion \(R:\mathbb{D}\to\mathbb{R}^m\) gibt, für die gilt: \(f(x)=f(x_0)+A(x-x_0)+\vert x-x_0 \vert\cdot R(x)\) und \[\lim_{x\to x_0} R(x)=0\]
\end{defi}

\begin{satz}[Jacobi-Matrix]
    Es sei \(f:\mathbb{D}\to \mathbb{R}^m,\mathbb{D}\subset\mathbb{R}^n\) eine Abbildung und \(x_0\in\mathbb{D}\). Weiterhin sei f in \(x_0\) total differenzierbar mit der Matrix
    \[A=(a_{ij});i=1,\dots,m;j=1,\dots,n \in\mathbb{R}^{m\times n}\]
    Dann ist f in \(x_0\) stetig und alle Komponentenfunktionen \(f_1,\dots,f_m:\mathbb{R}^n\to\mathbb{R}\) sind in \(x_0\) partiell differenzierbar, wobei gilt: \(a_{ij}=\pd{f_i}{x_j}(x_0)\).
    
    \[A=\begin{pmatrix}
    \dpd{f_1}{x_1}(x_0) & \dpd{f_1}{x_2}(x_0) & \cdots & \dpd{f_1}{x_n}(x_0) \\
    \dpd{f_2}{x_1}(x_0) & \dpd{f_2}{x_2}(x_0) & \cdots & \dpd{f_2}{x_n}(x_0) \\
    \vdots & \vdots & \ddots & \vdots \\
    \dpd{f_m}{x_1}(x_0) & \dpd{f_m}{x_2}(x_0) & \cdots & \dpd{f_m}{x_n}(x_0)
    \end{pmatrix}\]
    
    Die Matrix heißt Funktionalmatrix oder auch Jacobi-Matrix von f und wird mit \(Df(x_0)\) oder \(J_f(x_0)\) bezeichnet.
\end{satz}

\begin{beispiel}
    Gegeben sei die Funktion \(g:\mathbb{R}\to\mathbb{R}^2\) durch \[g(x)=\begin{pmatrix}2x^3 \\ \ln(x^2+1)
    \end{pmatrix} = \begin{pmatrix}
    g_1(x) \\ g_2(x)
    \end{pmatrix}\]
    Berechnen Sie die Jacobi-Matrix \(D_g(x)\):
    \[D_g(x)=\begin{pmatrix}6x^2 \\ \frac{1}{x^2+1}\cdot2x\end{pmatrix}\]
\end{beispiel}

\begin{aufg}
    \[f(x)=\begin{pmatrix}
    x_1 + x_2 \\
    x_1^2\cdot\sin x_2 \\
    e^{x_1x_2}
    \end{pmatrix}\]
    \[D_f(x_1,x_2)=?\]
\end{aufg}

\begin{bemerkung}
    Nach Definition 1.2.4 kann eine Funktion \(f(x)\), die total differenzierbar ist, in der Nähe von \(x_0\) durch \[f(x_0)+D_f(x_0)(x-x_0)\] angenähert werden.
    
    Da die Annäherungsfunktion linear ist, spricht man auch von Linearisierung.
    
    Hierfür muss aber klar sein, dass f auch wirklich total differenzierbar ist.
\end{bemerkung}

\begin{satz}
    Es sei \(f:\mathbb{D}\to \mathbb{R}^m,\mathbb{D}\subset\mathbb{R}^n\) eine Abbildung, deren Komponentenfunktionen \(f_1,\dots,f_m\) alle stetig partiell differenzierbar sind. Dann ist f total differenzierbar.
\end{satz}

\begin{beispiel}
    Gegeben sei \(f:\mathbb{R}^3\to\mathbb{R}\) durch
    \[f(x_1,x_2,x_3)=x_1^2x_2+x_1x_2\sin(x_3)\]
    \(Df(x_1,x_2,x_3)=\begin{pmatrix}
    2x_1x_2+x_2\sin(x_3) & x_1^2+x_1\sin(x_3) & x_1x_2\cos(x_3)
    \end{pmatrix}\)
    
    f soll in der Nähe des Punktes \(x_0=(x_{01},x_{02},x_{03})=(1,1,\frac{\pi}{2})\) angenähert werden. Zur Berechnung der Näherung wird benötigt: \(f(x_0)+Df(x_0)(x-x_0)\)
    
    \(f(x_0)=1+1\cdot1=2\)
    
    \(Df(x_0)=\begin{pmatrix}3 & 2 & 0\end{pmatrix}\)
    
    \(2+\begin{pmatrix}3 & 2 & 0\end{pmatrix}\cdot\begin{pmatrix}
    x_1 - 1 \\
    x_2 - 1 \\
    x_3 - \frac{\pi}{2}
    \end{pmatrix}\)
    
    \(f(x_1,x_2,x_3)=3x_1+2x_2-3\)
\end{beispiel}

\begin{aufg}
Gegeben sei die Funktion \(f:\mathbb{R}^2\to\mathbb{R}^3\) durch 

\(f(x_1,x_2)=\begin{pmatrix}x_1+x_2 & x_1^2\sin(x_2) & e^{x_1x_2}\end{pmatrix}^T\) (siehe Aufgabe 1.2.7)

    Berechnen Sie die Näherung der Funktion in der Nähe von \(x_0=(1,0)\)
\end{aufg}

\begin{aufg}
    Berechnen Sie die Tangentialebenen an die Funktion
    
    \(f(x_1,x_2)=e^{-2.5x_1^2-{(x_3-1)}^2}\) \((x_1,x_2)\in\mathbb{R}^2\) im Punkt \((x_{01},x_{02},f(x_{01},x_{02}))=(0,1,1)\).

    Verwenden Sie dazu zunächst die Formel für die Tangentialebenen und dann die Formel für die totale Differenzierbarkeit.
\end{aufg}

\begin{defi}[totale Differenzierbarkeit]
Sei \(f:\mathbb{R}^2\to\mathbb{R}\) eine total differenzierbare Funktion. Dann kann f in der Nähe eines Punktes \((x_{01},x_{02})\in\mathbb{R}^2\) durch \(f(x_{01},x_{02})+Df(x_{01},x_{02})\begin{pmatrix}x_1-x_{01} \\ x_2-x_{02}\end{pmatrix}\) angenähert werden.

    Es gilt also \(f(x_1,x_2)=f(x_{01},x_{02})+\begin{pmatrix}
        \dpd{f}{x_1}(x_{01},x_{02}) & \dpd{f}{x_2}(x_{01},x_{02})\end{pmatrix}\begin{pmatrix}
            x_1-x_{01} \\ x_2-x_{02}
        \end{pmatrix}\)

\[f(x_1,x_2)-f(x_{01},x_{02})\approx\dpd{f}{x_1}(x_{01},x_{02})\cdot\underbrace{(x_1-x_{01})}_{=:\Delta x_1} + \dpd{f}{x_2}(x_{01},x_{02})\cdot\underbrace{(x_2-x_{02})}_{=:\Delta x_2}\]

also gilt: \(\Delta f\approx \pd{f}{x_1}\cdot\Delta x_1 + \pd{f}{x_2}\cdot\Delta x_2\)

Für \enquote{beliebig kleines} \(\Delta x_1\) und \(\Delta x_2\) schreiben wir \enquote{d} statt \enquote{\(\Delta\)} und \enquote{=} statt \enquote{\(\approx\)}.

\[\boxed{df = \dpd{f}{x_1}(x_{01},x_{02})\cdot dx_1 + \dpd{f}{x_2}(x_{01},x_{02})\cdot dx_2} \]
Diesen Ausdruck bezeichnet man als \underline{totales Differenzial} der Funktion f.

Für eine Funktion \(f:\mathbb{R}^n\to\mathbb{R}\) gilt:
\[\boxed{df = \dpd{f}{x_1}\cdot dx_1 + \dpd{f}{x_2}\cdot dx_2 + \cdots + \dpd{f}{x_2}\cdot dx_n}\]
\end{defi}

\begin{bemerkung}
    Mit Hilfe des totalen Differentials kann der Einfluss der Änderung der Inputgrößen auf den Funktionswert abgeschätzt werden.
\end{bemerkung}

\begin{beispiel}
    Gegeben sei \(f:\mathbb{R}^2\to\mathbb{R}\) durch \(f(x_1,x_2)=x_1^2+\sin(x_1\cdot x_2)\). Wir betrachten f in der Nähe des Punktes \(f(x_{01},x_{02})=(\sqrt{\pi},\sqrt{\pi})\)

    Was passiert wenn wir leicht von dem Wert abweichen?

    \[\dpd{f}{x_1}=2x_1+\cos(x_1\cdot x_2)x_2\]
    \[\dpd{f}{x_2}=\cos(x_1 \cdot x_2)x_1\]

    \(df=2\sqrt{\pi}+\cos(\pi)\cdot\sqrt{\pi}\cdot(x_1-\sqrt{\pi})+\cos(\pi)\cdot\sqrt{\pi}\cdot(x_2-\sqrt{\pi}) \)
    \(df=\sqrt{\pi}\cdot dx_1+(-\sqrt{\pi})\cdot dx_2\)

    Veränderung der Inputgrößen z.B. \(dx_1=0.1\) \(dx_2=-0.1\)

    \(df=0.2\cdot\sqrt{\pi}\)
\end{beispiel}

\begin{bemerkung}
    Durch Einsetzen der Maximalen absoluten Fehler und Bilden der Beträge ergibt sich der (lineare) maximale absolute Fehler:

    \[\left|\Delta f\right|\approx \left|\dpd{f}{x_1}(x_1,\dots,x_n)\right|\cdot\left|\Delta x_1\right|+\cdots+\left|\dpd{f}{x_n}(x_1,\dots,x_n)\right|\cdot\left|\Delta x_n\right|\]
\end{bemerkung}

\begin{aufg}
    Das Volumen V eines geraden Kreiskegels wird berechnet durch: \(V=\frac{\pi}{3}\cdot r^2 \cdot \sqrt{k^2-r^2}\).

    r\dots Radius r=1m und absoluter Fehler von 0.01m

    k\dots Mantellinie k=1.5m und der absolute Fehler: 0.005m

    Berechnen Sie den (linearen) maximalen absoluten Fehler.

    Papula zu Abschnitt 2 S. 332: 12, 13, 15
\end{aufg}

\section{Extremwerte}

\begin{defi}[Lokales Extremum]
    Es seien \(U\subset\mathbb{R}^n\) eine offene Menge und \(f:U\to\mathbb{R}\) eine Abbildung.
    \begin{enumerate}[label=\emph{(\roman*)}]
        \item Ein Punkt \(x_0\in U\) heißt \underline{lokales Maximum} von \(f\), falls \(f\) in der Nähe von \(x_0\) nicht größer wird als bei \(x_0\), das heißt:
        \(f(x)\le f(x_0)\) für alle \(x\) in der Nähe von \(x_0\).
        \item Ein Punkt \(x_0\in U\) heißt \underline{lokales Minimum} von \(f\), falls \(f\) in der Nähe von \(x_0\) nicht kleiner wird als bei \(x_0\), das heißt:
        \(f(x)\ge f(x_0) \) für alle \(x\) in der Nähe von \(x_0\).
    \end{enumerate}
    Ein \underline{lokales Extremum} ist ein lokales Minimum oder ein lokales Maximum.
\end{defi}

\begin{bemerkung}
    Wie im eindimensionalen Fall liefert die Differentialrechnung nur Informationen über lokale und nicht über globale Extrema.
\end{bemerkung}

\begin{bemerkung}
    Es sei \(f:(a,b)\to\mathbb{R}\) differenzierbar in \(x_0\in (a,b)\) mit einem lokalen Extremum in \(x_0\). Dann ist \(f'(x_0)=0\).
\end{bemerkung}

\begin{satz}
    Es sei \(U \subset \mathbb{R}^n\) eine offene Menge und \(f:U\to\mathbb{R}\) differenzierbar. Besitzt f in \(x_0\in U\) ein lokales Extremum, so gilt:
    \[\nabla f(x_0)=0,\]
    d.h. \(\pd{f}{x_1}(x_0)=\cdots=\pd{f}{x_n}(x_0)=0\)
\end{satz}

\begin{beispiel}
    \(f:\mathbb{R}^2\to\mathbb{R}\)\hspace{1cm}\(f(x_1,x_2)=1-x_1^2-x_2^2\)
    
    \[\dpd{f}{x_1}(x_1,x_2)=-2x_1\hspace{1cm}\dpd{f}{x_2}(x_1,x_2)=-2x_2\]
    
    \(\to x_1=x_2=0\)
    
    Das heißt falls es ein Extremum gibt, dann liegt es bei (0,0).
\end{beispiel}

\begin{aufg}
    \[f:\mathbb{R}^2\to\mathbb{R}\hspace{1cm}f(x_1,x_2)=\sin(x_1)\cdot\sin(x_2)\]

    \[\nabla f(x_1,x_2)=\begin{pmatrix}
        \cos(x_1)\cdot\sin(x_2) \\
        \sin(x_1)\cdot\cos(x_2)
    \end{pmatrix} =: \begin{pmatrix}0 \\ 0\end{pmatrix}\]

        \(\cos(x_1)\cdot\sin(x_2)=0\)

        \((x_1=\pi)\) od. \(\sin(x_2)=0\)
\end{aufg}

\begin{bemerkung}
    Es sei \(f:(a,b)\to\mathbb{R}\) zweimal differenzierbar und \(f'(x_0)=0\).
    \begin{enumerate}[label=\emph{(\roman*)}]
        \item Ist \(f''(x_0)>0\), so hat f in \(x_0\) ein lokales Minimum.
        \item Ist \(f''(x_0)<0\), so hat f in \(x_0\) ein lokales Maximum.
    \end{enumerate}
\end{bemerkung}

\begin{defi}
    Eine Matrix \(A=(a_{ij}), i,j=1,\dots,n \in \mathbb{R}^{n\times n}\) heißt \underline{positiv definit}, falls gilt:


    \(a_{11}>0,\det\begin{pmatrix}
    a_{11} & a_{12} \\ a_{21} & a_{22}
    \end{pmatrix} > 0, \dots, \det\begin{pmatrix}
    a_{11} & \cdots & a_{1n} \\
    a_{21} & \cdots & a_{2n} \\
    \vdots & \ddots & \vdots \\
    a_{n1} & \cdots & a_{nn}
    \end{pmatrix} > 0\), 
    
    also \(\det\begin{pmatrix}
    a_{11} & \cdots & a_{1k} \\
    a_{21} & \cdots & a_{2k} \\
    \vdots & \ddots & \vdots \\
    a_{k1} & \cdots & a_{kk}
    \end{pmatrix} > 0\) für alle \(k=1,\dots,n\).
    
    A heißt \underline{negativ definit}, falls -A positiv definit ist.
\end{defi}

\begin{beispiel}
    Es sei \(A=\begin{pmatrix}
    1 & 1 \\ 1 & 4
    \end{pmatrix} \in \mathbb{R}^{2\times 2}\)
    
    \(\underline{k=1}:1>0\)
    
    \(\underline{k=2}: \det\begin{pmatrix}
    1 & 1 \\ 1 & 4
    \end{pmatrix} = 1\cdot4-1\cdot1=3>0\)
    
    \(\to\) A ist positiv definit.
\end{beispiel}

\begin{aufg}
    \(B=\begin{pmatrix}
    -6 & 2 \\ 2 & -1
    \end{pmatrix} \in \mathbb{R}^{2\times 2}\)
    
    \(-6<0 \to\) B nicht positiv definit
    
    \(-B=\begin{pmatrix}
    6 & -2 \\ -2 & 1
    \end{pmatrix}\)
    
    \(\underline{k=1}:6>0\)
    
    \(\underline{k=2}: \det\begin{pmatrix}
    6 & -2 \\ -2 & 1
    \end{pmatrix} = 6\cdot1 - (-2)\cdot(-2)=2>0\)
    
    \(\to\) B ist negativ definit.
\end{aufg}

\begin{aufg}
    Prüfen Sie \(C=\begin{pmatrix}
    1 & 1 & 0 \\ 1 & 3 & 1 \\ 0 & 1 & 2
    \end{pmatrix} \in \mathbb{R}^{3\times3}\) auf positive Definitheit.
\end{aufg}

\begin{bemerkung}
    \(D=\begin{pmatrix}
    0 & 1 \\ 1 & 4
    \end{pmatrix} \in \mathbb{D}^{2\times2}\) ist weder positiv noch negativ definit: \(d_{11}\not>0\to\) nicht positiv definit
    
    \(-d_{11}\not>0\to\) nicht negativ definit
\end{bemerkung}

\begin{defi}
    Es seien \(U\subset\mathbb{R}^n\) eine offene Menge, \(f:U\to\mathbb{R}\) eine zweimal stetig partiell differenzierbare Funktion und \(x_0\in U\). Unter der \underline{Hesse-Matrix} von f in \(x_0\) versteht man die Matrix: \[H_f(x_0)=\begin{pmatrix}
    \frac{\partial^2 f}{\partial x_1^2}(x_0) & \cdots & \frac{\partial^2 f}{\partial x_1 \partial x_n}(x_0) \\
    \frac{\partial^2 f}{\partial x_2 \partial x_1}(x_0) & \cdots & \frac{\partial^2 f}{\partial x_2 \partial x_n}(x_0) \\
    \vdots & \ddots & \vdots \\
    \frac{\partial^2 f}{\partial x_n \partial x_1}(x_0) & \cdots & \frac{\partial^2 f}{\partial x_n^2}(x_0) \\
    \end{pmatrix}\]
\end{defi}

\begin{beispiel}
    Gegeben sei \(f:\mathbb{R}^2\to\mathbb{R}\) durch \(f(x_1,x_2)=x_1^2+x_2^2\). Gesucht ist die Hesse-Matrix.
    
    \[H_f(x_1,x_2)=\begin{pmatrix}
    \frac{\partial^2 f}{\partial x_1^2}(x_1,x_2)  & \frac{\partial^2 f}{\partial x_1 \partial x_2}(x_1,x_2) \\
    \frac{\partial^2 f}{\partial x_2 \partial x_1}(x_1,x_2) & \frac{\partial^2 f}{\partial x_2^2}(x_1,x_2)
    \end{pmatrix}=\begin{pmatrix}
    2 & 0 \\ 0 & 2
    \end{pmatrix}\]
\end{beispiel}

\begin{aufg}
    Gegeben sei \(f:\mathbb{R}^3\to\mathbb{R}\) durch
    \[f(x_1,x_2,x_3)=x_1^3 \cdot x_2^3 \cdot \sin(x_3)\]
    Berechnen Sie die Hesse-Matrix \(H_f(0,0,0)\) \& \(H_f(1,1,0)\).
\end{aufg}

\begin{satz}
    Es sei \(U\subset\mathbb{R}^n\) eine offene Menge, \(f:U\to\mathbb{R}\) zweimal stetig differenzierbar und \(x_0\in U\) ein Punkt mit \(\nabla f(x_0)=0\).
    \begin{enumerate}[label=\emph{(\roman*)}]
        \item Ist \(H_f(x_0)\) positiv definit, so hat f in \(x_0\) ein lokales Minimum.
        \item Ist \(H_f(x_0)\) negativ definit, so hat f in \(x_0\) ein lokales Maximum.
        \item Ist \(\underline{U\subset\mathbb{R}^2}\) und gilt \(\det H_f(x_0) < 0\), so liegt kein Extremwert vor.
    \end{enumerate}
\end{satz}

\begin{beispiel}
    Gegeben sei \(f:\mathbb{R}^2\to\mathbb{R}\) durch \(f(x_1,x_2)=1+x_1^2+x_2^2\)
    
    \(\nabla f(x_1,x_2)=\begin{pmatrix}
    2x_1 \\ 2x_2
    \end{pmatrix} := 0 \to x_1=x_2=0\)
    
    \(\to (0,0)\) könnte ein Extremum sein
    
    Überprüfen durch Hesse-Matrix

    \(H_f(x_1,x_2)=\begin{pmatrix}
    2 & 0 \\ 0 & 2
    \end{pmatrix}\)

    \(2>0, \det\begin{pmatrix}
    2 & 0 \\ 0 & 2
    \end{pmatrix}=4>0\)
    
    \(\to H_f(0,0)\) positiv definit \(\to\) in \((0,0)\) liegt ein lokales Minimum vor.
\end{beispiel}

\begin{aufg}
    Gegeben sei \(f:\mathbb{R}^2\to\mathbb{R}\) durch \(f(x_1,x_2)=\sin(x_1)\cdot\sin(x_2)\)
    
    Untersuchen Sie die Funktion auf lokale Extrema.
    
    Kandidaten: \((0,0),\left(\frac{\pi}{2},\frac{\pi}{2}\right)\)
\end{aufg}

\begin{aufg}
    Gegeben sei die Funktion \(f:\mathbb{R}^2\to\mathbb{R}\) durch \(f(x_1,x_2)=\cos(x_1)+\cos(x_2)\)
    
    Gradient: \(\nabla f(x_1,x_2)=\begin{pmatrix}
    -\sin x_1 \\ -\sin x_2
    \end{pmatrix} := \begin{pmatrix}
    0 \\ 0
    \end{pmatrix} \to -\sin x_1 = -\sin x_2 = 0\)
    
    \((k_1\pi,k_2\pi), k_1,k_2\in\mathbb{Z}\)
    
    Hesse-Matrix
    
    \(H_f(x_1,x_2)=\begin{pmatrix}
    -\cos x_1 & 0 \\ 0 & -\cos x_2
    \end{pmatrix}\)
    
    \(H_f(k_1\pi,k_2\pi)=\begin{pmatrix}
    -{(-1)}^{k_1} & 0 \\ 0 & -{(-1)}^{k_2} 
    \end{pmatrix}\)
    
    \vspace{1cm}
    \def\arraystretch{1.25}	
    \begin{tabular}{|c|c|c|} \hline
    & \(k_1\) gerade & \(k_1\) ungerade \\ \hline
    \(k_2\) gerade & lokales Maximum & kein Extremwert \\ \hline
    \(k_2\) ungerade & kein Extremwert & lokales Minimum \\ \hline
    \end{tabular}
\end{aufg}

\begin{beispiel}[Nebenbedingungen]
    Gegeben sei 12m langer Draht, aus dem die Kanten eines Quaders von möglichst großem Volumen hergestellt werden sollen. Gesucht sind die Kantenlängen \(x_1,x_2,x_3\) des optimalen Quaders.
    
    \(4x_1+4x_2+4x_3=4(x_1+x_2+x_3)=12\)
    
    \(x_1+x_2+x_3=3\)
    
    \(V=x_1\cdot x_2 \cdot x_3\)
    
    \(x_1,x_2,x_3 > 0\)
    
    \(V=x_1 x_2 (3-x_1-x_2)\)
    
    \(V=3x_1x_2-x_1^2x_2-x_1x_2^2\)
    
    \(\mathbb{D}={(x_1,x_2)\in\mathbb{R}^2:x_1>0,x_2>0,x_1+x_2<3}\), \(\mathbb{D}\) ist eine offene Menge
    
    \(\nabla V(x_1,x_2)=\begin{pmatrix}
    3x_2-2x_1x_2-x_2^2 \\
    3x_1-x_1^2-2x_1x_2
    \end{pmatrix} := \begin{pmatrix}
    0 \\ 0
    \end{pmatrix}\)
    
    \(\to x_1=1, x_2=1, x_3=1\)
    
    Berechnen Sie die Hesse-Matrix

    Beispiel ***:
    
    \[f:\mathbb{R}^2\to\mathbb{R} f(x,y)=\begin{cases}
        x \cdot y\cdot\frac{x^2-y^2}{x^2+y^2} & (x,y)\ne(0,0) \\
        0 & (x,y)=(0,0)
    \end{cases}\]

    (a) Stetigkeit in (0,0)

    \[\bar{P} \lim_{P\to\bar{P}}f(x_1,\dots,x_n)=f(\bar{x_1},\dots,\bar{x_n})\]

    Es gilt: \(\lim_{(x,y)\to(0,0)} f(x,y)=\lim_{(x,y)\to(0,0)} |x\cdot y\cdot\frac{x^2-y^2}{x^2+y^2}|=\lim_{(x,y)\to(0,0)} \underbrace{|x\cdot y|}_{\to 0}\cdot\underbrace{|\frac{x^2-y^2}{x^2+y^2}|}_{\leq 1}\)
\end{beispiel}

\begin{aufg}
    Papula S.332 zu Abschnitt 2 \(\to\) Aufg. 24
    
    \begin{enumerate}[label=\emph{(\roman*)}]
        \item \(f(x_1,x_2)=x_1^2(1-x_2)-x_2^3+12x_2+13\)
        \item \(f(x_1,x_2)={(x_1-1)}^2(1-x_2)-x_2^3+12x_2+3\)
        \item \(f(x_1,x_2)=4(x_1^2-25)(x_2-2)+5x_2^2+12x_2\)
    \end{enumerate}
\end{aufg}

\chapter{Grundlagen der Integralrechnung reeller Funktionenmit mehreren Variablen}

    \section{Zweidimensionale Integralrechnung}

    \begin{defi}[beschränkt]
        Eine Menge \(U\subset\mathbb{R}^2\) heißt \underline{beschränkt}, wenn es ein Rechteck R gibt, sodass \(U\subset R\) gilt.
    \end{defi}

    \begin{bemerkung}
        %\begin{tikzpicture}
        %	\begin{axis}
        %		\addplot3[] coordinates {
        %			(0,0,0) (0,0.5,0) (0,1,0) (0,1.5,0) (0,2,0)
        %			(1,0,0) (1,0.5,0) (1,1,0) (1,1.5,0) (1,2,0)
        %			(2,0,0) (2,0.5,0) (2,1,0) (2,1.5,0) (2,2,0)
        %		}
        %	\end{axis}
        %\end{tikzpicture}
    \end{bemerkung}
    
    \begin{bemerkung}
        Es sei \(U\subset \mathbb{R}^2\) eine beschränkte Menge und \(f:U\to \mathbb{R}\) eine stetige Funktion. Da ein Volumen betrachtet wird, werden die alten Näherungsrechtecke durch Näherungsquader ersetzt und deren Volumen zusammengezählt.

        U wird also in n kleine Teilbereiche \(u_1,\dots,u_n\) zerlegt. Die Fläche dieser Teilbereiche wird mit \(\delta u_1,\dots,\Delta u_n\) bezeichnet.

        Zur Berechnung des Rauminhalts des Quaders wird weiterhin die Quaderhöhe benötigt. Dazu wird ein Punkt \((x_i,y_i)\in U_i\) gewählt und sein Funktionswert \(f(x_i,y_i)\) als Höhe des Quaders betrachtet. Das Teilvolumen beträgt dann \(f(x_i,y_i)\cdot\Delta u_i\)

        Falls \(U_i\) ein Rechteck ist mit den Seiten \(\Delta x_i\) und \(\Delta y_i\), so ergibt sich das Teilvolumen
        
        \[\Delta V_i = f(x_i,y_i)\cdot \Delta U_i=f(x_i,y_i)\cdot \Delta x_i\cdot \Delta y_i\]

        Als Näherung für das Gesamtvolumen eribt sich also

        \[V_n(f)=\sum^{n}_{i=1} f(x_i,y_i)\cdot\Delta U_i\]

        Hier wurde der gesamte Definitionsbereich U in die n Teilbereiche \(U_1,\dots,U_n\) zerlegt.

        Der genaue Wert für das Volumen kann berechnet werden, indem n gegen \(\infty\) geht.

        Deshalb wird definiert:

        \[\int_{U} f(x,y)dU=\int_{U} f(x,y)dxdy\]

        \[= \lim_{n\to\infty} \sum^{n}_{i=1} f(x_i,y_i)\cdot\Delta U_i\]

        Um klarzustellen, dass es sich um ein zweidimensionales Integral handelt, werden oft die zwei Integralsymbole verwendet:

        \[\iint_{U} f(x,y)dU=\iint_{U} f(x,y)dxdy\]
    \end{bemerkung}

    \begin{defi}[konvex]
        Eine Menge \(U\subset\mathbb{R}^2\) oder \(U\subset\mathbb{R}^3\) heißt \underline{konvex}, falls für alle Punkte \(x,y\in U\) auch die gesamte Verbindungsstrecke von x nach y in U liegt.
    \end{defi}

    \begin{bemerkung}
        % Skizze

        Es sei \(U\subset\mathbb{R}^2\) eine beschränkte und konvexe Menge und \(f:U\to\mathbb{R}\) eine stetige Funktion. Da U beschränkt ist, gibt es einen kleinsten vorkommenden x-Wert a und einen größten vorkommenden x-Wert b.

        Der Flächeninhalt der Schnittfläche des Körpers bei einem beliebigen x-Wert zwischen a und b wird mit \(I(x)\) bezeichnet.

        Der Körper, dessen Volumen wir ausrechnen, setzt sich aus all diesen Schnittflächen zusammen. Für jedes \(x\in [a,b]\) existiert eine Schnittfläche mit Flächeninhalt \(I(x)\), das heißt durch Aufsummieren dieser unendlich vielen Flächeninhalte ergibt sich das Volumen des Körpers.

        \[\iint_U f(x,y)dxdy = \int^a_b I(x)dx\]

        Der Flächeninhalt von \(I(x)\) kann einfach mit einem eindimensionalen Integral berechnet werden.

        \[I(x) = \int^{y_o(x)}_{y_u(x)} f(x,y)dy \]

        Insgesamt ergibt sich also:

        \[\iint_U f(x,y)dxdy = \int^b_a \left(\int^{y_o(x)}_{y_u(x)} f(x,y)dy\right)dx\]
    \end{bemerkung}

    \begin{satz}
        Es sei \(U\subset\mathbb{R}^2\) eine beschränkte und konvexe Menge und \(f:U\to\mathbb{R}\) eine stetige Funktion. Weiterhin sei a der kleinste in U vorkommende x-Wert und b der größete in U vorkommende x-Wert. Für \(x\in [a,b]\) bezeichnen wir den kleinsten y-Wert für den \((x,y) \in U\) gilt, als \(y_u(x)\) und den größten y-Wert als \(y_o(x)\).

        Dann ist \(\iint_U f(x,y)dxdy = \int^b_a \left(\int^{y_o(x)}_{y_u(x)} f(x,y)dy\right)dx\)
    \end{satz}

    \begin{beispiel}
        \begin{enumerate}[label=\emph{(\roman*)}]

            \item \(U=\{(x,y):x \in [0,1], y \in [0,2]\}\) und \(f:U\to\mathbb{R}\) mit \(f(x,y)=x^2+y^2\)

                \(a=0,b=1,y_u(x)=0,y_o(x)=2\)

                \[\int^1_0 \left(\int^{2}_{0} (x^2+y^2)dy\right)dx\]
                \[\int^1_0 \left({\left[x^2y+\frac{y^3}{3}\right]}^{2}_{0}\right)dx\]
                \[\int^1_0 \left(2x^2+\frac{2^3}{3}\right)dx\]
                \[{\left[\frac{2}{3}x^3+\frac{8}{3}x\right]}^1_0\]
                \[\frac{2}{3}+\frac{8}{3}=\frac{10}{3}\]

        \item \(U=\{(x,y):x \geq 0, x \leq 1, y \leq x, y \geq 0\}\)

            \(f:U\to\mathbb{R}\) mit \(f(x,y)=x\cdot\sin(y)\)

            \(a=0,b=1,y_u(x)=0,y_o(x)=x\)

            \[\int^1_0 \left(\int^{x}_{0} x\cdot\sin(y)dy\right)dx\]
            \[\int^1_0 \left({\left[x\cdot-\cos(y)\right]}^{x}_{0}\right)dx\]
            \[-\int^1_0 \left(x\cdot\cos(x)+x\right)dx\]
            \[-{\left[x\cdot\sin(x)+\cos(x)+\frac{x^2}{2}\right]}^1_0\]
            \[-\left(1\cdot\sin(1)+\cos(1)+\frac{1^2}{2}-\cos(0)\right)=-\sin(1)-\cos(1)+\frac{3}{2}\]

        \item \(U=\left\{(x,y):x^2+y^2\le1\right\}\)
        
        \(f:U\to\mathbb{R},f(x,y)=1\)

        \begin{tikzpicture}
            \begin{axis}[
                ymin=-2,
                ymax=2,
                xmin=-2,
                xlabel=x,
                ylabel=y,
                xmax=2,
                axis lines = middle,
                axis equal,
            ]
                \draw (axis cs:0,0) circle [blue, radius=1];
            \end{axis}
            
        \end{tikzpicture}

        \[\iint_U\left(1\right)dxdy=\int_{-1}^1 \left(\int_{-\sqrt{1-x^2}}^{\sqrt{1-x^2}}1dy\right)dx\]
        \[\int_{-1}^1 \left({\left[y\right]}_{-\sqrt{1-x^2}}^{\sqrt{1-x^2}}\right)dx\]
        \[2\int_{-1}^1 \sqrt{1-x^2}dx\]

        \begin{einschub}
            Substitution: \(f(x)=\sqrt{1-x^2}\)

            \(\int f(x(t))\cdot x'(t)dt=\int f(x)dx\)
            
            \(x(t)=\sin(t) \to t=\arcsin(x)\)

            \(\int f(x)dx=\int \sqrt{1-x^2}dx=\int f(\sin(t))\cdot x'(t)dt=\int \sqrt{1-\sin^2(t)}\cdot\cos(t)dt\)

            \(=\int \cos(t)\cdot\cos(t)dt=\cos(t)\cdot\sin(t)+\int \sin(t)\cdot\sin(t)dt=\int \sin^2(t)dt=\int 1-\cos^2(t)dt\)

            \(=\cos(t)\cdot\sin(t)+t-\int \cos^2(t)dt\)

            \(\to \int \cos^2(t)dt=\frac{1}{2}\sin(t)\cdot\cos(t)+\frac{t}{2}+C\)

            Rücksubstitution

            \(\frac{x}{2}\cdot\cos(t)+\frac{t}{2}+C\)

            \(\cos(t)=\cos(\arcsin(x))=\sqrt{1-\sin^2(\arcsin(x))}=\sqrt{1-x^2}\)

            \(\frac{x}{2}\cdot\sqrt{1-x^2}+\frac{\arcsin(x)}{2}+C\)

            \(\int \sqrt{1-x^2}dx=\frac{x}{2}\sqrt{1-x^2}+\frac{\arcsin(x)}{2}\)
        \end{einschub}
        \[2\int_{-1}^1 \sqrt{1-x^2}dx={\left[x\sqrt{1-x^2}+\arcsin(x)\right]}^1_{-1}\]
        \[\arcsin(1)-\arcsin(-1)=\frac{\pi}{2}+\frac{\pi}{2}=\pi\]
        \end{enumerate}
    \end{beispiel}

    \begin{satz}
        Es sei \(U\subset\mathbb{R}^2\) eine beschränkte und konvexe Menge und \(f:U\to\mathbb{R}\) eine stetige Funktion. Weiterhin sei a der kleinste in U vorkommende y-Wert und b der größte in U vorkommende y-Wert.

        Für \(y\in[a;b]\) bezeichnen wir den kleinsten x-Wert, für den \((x,y)\in U\) gilt als \(x_u(y)\) und den größten als \(x_o(y)\).

        Dann ist \(\iint_U f(x,y)dxdy=\int_a^b \left(\int_{x_u(y)}^{x_o(y)} f(x,y)dx\right)dy\)
    \end{satz}

    \begin{beispiel}
        \[\iint_U x^2+y^2dxdy\] \(U=\left[0,1\right]\times\left[0,2\right]\)

        \begin{tikzpicture}
            \begin{axis}
                [
                    xlabel=x,
                    ylabel=y,
                    xmin=0,
                    ymin=0,
                    ymax=2.4,
                    xmax=2,
                ]
                \addplot[mark=none] (0,2) rectangle (1,0);
                \draw (0,1) -- (1.5,1);
            \end{axis}
        \end{tikzpicture}
    \end{beispiel}

    \begin{aufg}
        \(f:U\to\mathbb{R}\) mit \(f(x,y)=x^2+y\)

        \begin{enumerate}[label=\emph{(\roman*)}]
            \item \(U=\{(x,y):1 \leq x \leq 2, 2 \leq y \leq 3\}\)
            
            \begin{tikzpicture}
                \begin{axis}
                    [
                        xlabel=x,
                        ylabel=y,
                        xmin=0,
                        ymin=0,
                        ymax=3.5,
                        xmax=3,
                        width=5cm,
                    ]
                    \addplot[mark=none] (1,3) rectangle (2,2);
                \end{axis}
            \end{tikzpicture}

            \(\iint_U f(x,y)dxdy=\int_1^2 \left(\int_2^3 (x^2+y) dy\right)dx = \frac{29}{6}\)

            \item \(U=\{(x,y):0 \leq x \leq 1, x^2 \leq y \leq \sqrt{x}\}\)
            
            \begin{tikzpicture}
                \begin{axis}
                    [
                        xlabel=x,
                        ylabel=y,
                        xmin=0,
                        ymin=0,
                        ymax=1,
                        xmax=1,
                        width=5cm,
                    ]
                    \addplot[domain=0:1,mark=none,samples=200] {x^2};
                    \addplot[domain=0:1,mark=none,samples=200] {sqrt(x)};
                \end{axis}
            \end{tikzpicture}

            \(\iint_U f(x,y)dxdy=\int_0^1 \left(\int_{x^2}^{\sqrt{x}} (x^2+y) dy\right)dx = \frac{33}{140}\)
        \end{enumerate}
    \end{aufg}

    \begin{einschub}[Terassenpunkte]
        Jeder Punkt \(x_0\in D_f\) (Definitionsbereich) einer Funktion \(f:D_f\subset\mathbb{R}^n\to\mathbb{R}\) mit \(\nabla f(x_0)=0\) heißt \underline{kritischer Punkt} von f.

        Jeder kritische Punkt, von f, der nicht gleichzeitig ein lokales Extremum ist, heißt \underline{Terassenpunkt} von f.

        Beispiel:

        \(f(x,y)=x^2-y^2\)	\(\pd{f}{x}(x,y)=2x\)	\(\pd{f}{y}(x,y)=-2y\)

        \(\to (0,0)\) ist Kandidat

        Hessematrix: \(H_f(x,y)=\begin{pmatrix}2 & 0 \\ 0 & -2\end{pmatrix}\) \(\begin{vmatrix}2 & 0 \\ 0 & -2\end{vmatrix}\underline{\underline{-4<0}}\)

        \begin{tikzpicture}
            \begin{axis}
                [
                    xlabel=x,
                    ylabel=y,
                    zlabel=z,
                ]
                \addplot3[surf] {x^2-y^2};
            \end{axis}
        \end{tikzpicture}
    \end{einschub}
    
    \begin{aufg}
        Berechnen Sie den Flächeninhalt des Dreiecks mit den Eckpunkten \((2,2),(0,3),(1,0)\)

        \begin{tikzpicture}
            \begin{axis}
                [
                    xlabel=x,
                    ylabel=y,
                ]
                \addplot coordinates {(2,2) (0,3) (1,0) (2,2)};
                \draw (1,4) -- (1,-1);
            \end{axis}
        \end{tikzpicture}

        \(f_1(x) = -3x+3\)

        \(f_2(x) = 2x-2\)

        \(f_3(x) = -\frac{1}{2}x+3\)

        \(U_1=\left\{(x,y): 0 \le x \le 1, f_1(x) \le y \le f_3(x) \right\}\)

        \(U_2=\left\{(x,y): 1 \le x \le 2, f_2(x) \le y \le f_3(x) \right\}\)

        \(\int_0^1\left(\int_{f_1(x)}^{f_3(x)} 1dy\right)dx + \int_1^2\left(\int_{f_2(x)}^{f_3(x)} 1dy\right)dx\)
    \end{aufg}

    \begin{aufg}
        Berechnen Sie den Inhalt der Fläche, die von den Funktionen \(f_1(x)=x^2-1\) und \(f_2(x)=3x+3\) begrenzt wird.
        \\
        \begin{tikzpicture}
            \begin{axis}
                [
                    xlabel=x,
                    ylabel=y,
                    xmin=-1.5,
                    xmax=4.5,
                    ymax=16,
                    ymin=-2,
                ]
                \addplot[mark=none,samples=200] {(x^2)-1};
                \addplot[mark=none,samples=200] {3*x+3};
            \end{axis}
        \end{tikzpicture}

        \(\int_{-1}^4 \left(\int_{x^2-1}^{3x+3} 1dy\right)dx\)
    \end{aufg}

    \begin{aufg}
        Berechnen Sie \(\int_0^\infty x_1 x_2 \cdot e^{-x_1 x_2}dx_1,\ x_2>0\)
        \[\lim_{c\to\infty} \int_0^C x_1 x_2 \cdot e^{-x_1 x_2}dx_1\]
        \[\lim_{c\to\infty} \left(e^{-cx_2}\left(-c-\frac{1}{x_2}\right)+\frac{1}{x_2} \right)\]
        \[\lim_{c\to\infty} -\frac{c}{e^{cx_2}}=\lim_{c\to\infty} -\frac{1}{x_2e^{cx_2}}=0 \]
        \[\int_0^\infty x_1 x_2 \cdot e^{-x_1 x_2}dx_1=\frac{1}{x_2}\]
    \end{aufg}

    \section{Dreidimensionale Integralrechnung}

    \begin{satz}
        Es seien \(U\subset\mathbb{R}^3\) eine beschränkte und konvexe Menge und \(f:U\to\mathbb{R}\) eine stetige Funktion. Weiterhin sei a der in U kleinste vorkommende x-Wert und b der größte.

        Für \(x\in[a;b]\) bezeichnen wir den kleinsten y-Wert, für den es ein z gibt, sodass \((x,y,z)\in U\) gilt, mit \(y_u(x)\), und den größten mit \(y_o(x)\).

        Schließlich bezeichnen wir für zulässiges (x,y) mit \(z_u(x,y)\) den kleinsten z-Wert, sodass \((x,y,z)\in U\), und mit \(z_o(x,y)\) den größten z-Wert. Dann ist
        \[\iiint_U f(x,y,z)dxdydz=\int_a^b \left(\int_{y_u(x)}^{y_o(x)} \left(\int_{z_u(x,y)}^{z_o(x,y)} f(x,y,z)dz\right)dy\right)dx\]
    \end{satz}

    \begin{beispiel}
        \begin{enumerate}[label=\emph{(\roman*)}]
            \item \(U=[-1;1]\times[0,1]\times[0,2]\)
            
            \(f(x,y,z)=x^2+y^2+z^2\)

            \(int_{-1}^1 \left(\int_0^1 \left(\int_0^2 x^2+y^2+z^2dz \right)dy \right)dx\)

            \item \(U=\{(x,y,z):x,y,z \ge 0,x \le 1,y \le x,z \le y\}\)
            
            \(f(x,y,z)=x\cdot y^2 \cdot z\)
            
            \begin{tikzpicture}
                \begin{axis}
                    [
                        xlabel=x,
                        ylabel=y,
                        zlabel=z,
                        width=10cm,
                        zmin=0,
                        zmax=1,
                    ]
                    \addplot3[fill=blue!80,opacity=0.8] coordinates {(0,0,0) (1,1,0) (1,0,0) (0,0,0)};
                    
                    \addplot3[fill=red!80,opacity=0.8] coordinates {(0,0,0) (1,1,0) (1,1,1) (0,0,0)};
                    \addplot3[fill=orange!50!black,opacity=0.7] coordinates {(1,0,0) (1,1,0) (1,1,1) (1,0,0)};
                    \addplot3[fill=red!80,opacity=0.8] coordinates {(0,0,0) (1,1,1) (1,0,0) (0,0,0)};
                \end{axis}
            \end{tikzpicture}
            \(\int_0^1 \left(\int_0^x \left(\int_0^y xy^2zdz \right)dy \right)dx=\frac{1}{70}\)
        \end{enumerate}
    \end{beispiel}
    \begin{einschub}[Geometrische Betrachtungen zur Tangentialebene]
        Die Rolle, \\
        die die Kurventangente bei einer Funktion von einer Variablen spielt, übernimmt die sogenannte Tangentialebene bei einer Funktion von zwei Variablen \(z=f(x,y)\). Sie enthält sämtliche im Flächenpunkt \(P=(x_0,y_0,z_0)\) an die Bildfläche von \(z=f(x,y)\) angelegten Tangenten. In der unmittelbaren Umgebung ihres Berührungspunktes P besitzen Fläche und Tangentialebene im Allgemeinen keinen weiteren gemeinsamen Punkt.

        Herleitung der Funktionsgleichung dieser Tangentialebene in der Form: \(z=ax+by+c\)

        Die unbekannten Koeffizienten \(a,b,c\) werden aus den bekannten Eigenschaften der Tangentialebene bestimmt.

        Fläche und Tangentialebene besitzen im Berührungspunkt P die gleiche Steigung. Das bedeutet, dass dort die entsprechenden partiellen Ableitungen erster Ordnung über\-ein\-stimmen müssen. Die partiellen Ableitungen der Tangentialebene sind \(z_x(x,y)=a\) und \(z_y(x,y)=b\), die der Funktion \(z=f(x,y)\) lauten \(z_x(x,y)=f_x(x,y)\) und \(z_y(x,y)=f_y(x,y)\). An der Berührungsstelle \((x_0,y_0)\) gilt demnach: \(a=f_x(x_0,y_o)\) und \(b=f_y(x_0,y_0)\).

        Somit sind die Koeffizienten a und b bestimmt.

        Außerdem ist P ein gemeinsamer Punkt von Fläche und Tangentialebene: \(z_0=ax_0+by_0+c \to c=z_0-ax_0-by_0\)

        Einsetzen in die Gleichung für die Tangentialebene:

        \(z=ax+by+z_0-ax_0-by_0 = a(x-x_0)+b(y-y_0)+z_0=f_x(x_0,y_0)(x-x_0)+f_y(x_0,y_0)(y-y_0)+f(x_0,y_0)\)
    \end{einschub}

    \chapter{Differentialgleichungen}
    \section{Einführung}

    \begin{defi}[Gewöhnliche Differentialgleichungen]
        Eine Gleichung der
        \\
        Form \(F(x,y,y',y'',\dots,y^{(n)})=0\) für eine unbekannte Funktion \(y=f(x)\) und deren Ableitungen heißt \underline{gewöhnliche Differentialgleichung n-ter Ordnung}.
    \end{defi}

    \begin{bemerkung}
        Neben gewöhnlichen Differentialgleichungen gibt es auch partielle Differentialgleichungen, diese werden aber in dieser Vorlesung nicht behandelt.
    \end{bemerkung}

    \begin{beispiel}
        Betrachtet wird eine elastische Feder in Gleichgewichtslage. Wird an den Punkt \(P_0\) ein Körper der Masse m angehängt, so hat die Feder zum Zeitpunkt t eine gewisse Auslenkung \(y(t)\). Unter Vernachlässigung der Reibung gilt für die Rückstellkraft F der Feder, die auf die Masse m wirkt: \(F=-c \cdot y(t)\). Dabei bezeichnet c die Federkonstante.

        Wegen \(F=m \cdot y''(t)\) gilt: \(m \cdot y''(t)=-c \cdot y(t)\) also \(m \cdot y''(t) + c \cdot y(t) = 0\).

        \[\int\frac{dy}{y}=\int xdx \to \ln\left|y\right| = \frac{x^2}{2}+c \]
        \[\to |y(x)| = e^{\frac{x^2}{2}+c}=e^{\frac{x^2}{2}}\cdot \underbrace{e^c}_{c_1}\ ,c_1\in\mathbb{R}\]
    \end{beispiel}

    \section{Trennung der Variablen}
    \begin{beispiel}
        Gegeben sei die Gleichung \(y'=xy\).

        Gesucht ist die Funktion \(y(x)\).

        \(y'=\frac{dy}{dx} \to \frac{dx}{dy}=xy \to_{y \ne 0} \frac{dy}{y}=xdx\)
    \end{beispiel}

    \begin{defi}[Differentialgleichungen mit getrennten Variablen]
        Es seien \\ \(\mathcal{I},\mathcal{J}\subset\mathbb{R}\) offene Intervalle, \(f:\mathcal{I}\to\mathbb{R}\) und \(g:\mathcal{J}\to\mathbb{R}\) stetige Funktionen und \(g(y)\ne0\) für alle \(y\in\mathcal{J}\). Dann heißt die Differentialgleichung \(y'=f(x)\cdot g(y)\) eine Differentialgleichung mit getrennten Variablen.

        Auf diese Differentialgleichung lässt sich das gleiche Verfahren anwenden wie in Beispiel 3.2.1. Es kann eine allgemeine Lösungsformel abgeleitet werden.
    \end{defi}

    \begin{einschub}[Klausurtestaufgabe]
        Gegeben seien die Funktionen f,g durch \(g(x,y)=\begin{pmatrix}
            x+y \\ x^2-y \\ 2xy
        \end{pmatrix},\ f(x,y,z)=\begin{pmatrix}
            \sin(x) \\ \sin(y+z)
        \end{pmatrix}\)

        Berechnen Sie die Jacobi-Matrix von \(f\circ g\) mit Hilfe der Kettenregel: \(D(f\circ g)(x,y)=Df(g(x,y))\cdot Dg(x,y)\).

        \underline{Hinweis}: Zur Berechnung von \(Df(g(x,y))\) berechnen Sie zunächst \(Df(x,y,z)\) und setzen Sie anschließend den Ergebnisvektor von \(g(x,y)\) ein.

        \[Df(x,y,z)=\begin{pmatrix}
            \cos(x) & 0 & 0 \\
            0 & \cos(y+z) & \cos(y+z) \\\end{pmatrix}\]
        \[Df(x+y,x^2-y,2xy)=\begin{pmatrix}
                \cos(x+y) & 0 & 0 \\
                0 & \cos(x^2-y+2xy) & \cos(x^2-y+2xy) \\\end{pmatrix}\]
        \[Dg(x,y)=\begin{pmatrix}
            1 & 1 \\
            2x & -1 \\
            2y & 2x\end{pmatrix}\]
        \[D(f\circ g)(x,y)=\begin{pmatrix}
            \cos(x+y) & \cos(x+y) \\
            2\cos\alpha(x+y) & \cos\alpha(2x-1) \\
        \end{pmatrix}\]

        Papula S. 519 zu Abschnitt 1 Aufg. 1; S. 520 zu Abschnitt 2 Aufg. 4a+d,5a+c
    \end{einschub}
    \stepcounter{section}
    \stepcounter{section}
    \section{Lineare Differentialgleichungen}
    \begin{defi}[Homogene lineare Differentialgleichungen n-ter Ordnung]
        Es seien \(\mathcal{I} \subset \mathbb{R}\) ein Intervall und \(a_0,a_1,\dots,a_{n-1};\mathcal{I} \to \mathbb{R}\) stetige Funktionen. Die Gleichung
        \[y^{(n)}(x) + a_{n-1}(x) \cdot y^{(n-1)}(x) + \cdots + a_1(x) \cdot y'(x) + a_0(x) \cdot y(x) = 0 \]
        heißt \underline{homogene lineare Differentialgleichung n-ter Ordnung}.

        Ist weiterhin \(b:\mathcal{I}\to\mathbb{R}\) eine stetige Funktion, so heißt die Gleichung
        \[y^{(n)}(x) + a_{n-1}(x) \cdot y^{(n-1)}(x) + \cdots + a_1(x) \cdot y'(x) + a_0(x) \cdot y(x) = b(x) \]
        \underline{inhomogene lineare Differentialgleichung n-ter Ordnung}.
    \end{defi}

    \begin{defi}[linear unabhängig]
        Eine Menge von Funktionen \(\{y_1,\dots,y_n\}\) heißt \underline{linear unabhängig}, wenn man keine Funktion aus den anderen linear kombinieren kann, d.h.\ für eine beliebige Funktion \(y_i\) gibt es keine Kombination der Form \(y_i(x)=c_1y_1(x)+\cdots+c_{i-1}y_{i-1}(x)+c_{i+1}y_{i+1}(x)+\cdots+c_n y_n(x)\) mit \(c_1,\dots,c_n \in \mathbb{R}\).

        Falls man ein \(y_i\) aus den anderen Funktionen linear kombinieren kann, heißt die Menge \underline{linear abhängig}.
    \end{defi}

    \begin{beispiel}
        Es sei \(n=2,y_1(x)=\sin(x),y_2(x)=17\sin(x)\)

        Dann ist \(\{y_1,y_2\}\) linear abhängig, da für \(c=17\) gilt: \(y_2=cy_1\).
    \end{beispiel}

    \begin{satz}[Wronski-Determinante]
        Es sei \(L_H\) die Lösungsmenge einer linearen homogenen Differentialgleichung n-ter Ordnung. Dann gibt es n linear unabhängige Lösungen \(y_1,\dots,y_n\) der Differentialgleichung und es gilt: \(L_H=\{c_1y_1(x)+\cdots+c_n y_n(x)\}|c_1,\dots,c_n\in\mathbb{R}\).

        Aus den Grundlösungen \(y_1,\dots,y_n\) lässt sich also mit Hilfe von Linearkombinationen die gesamte Lösungsmenge berechnen.

        Weiterhin sind n Lösungen \(y_1,y_n \in L_H\) genau dann linear unabhängig, wenn für die \underline{Wronski-Determinante} folgendes gilt:
        \[W(x)=\det\begin{pmatrix}
            y_1(x) & \cdots & y_n(x) \\
            y_1'(x) & \cdots & y_n'(x) \\
            \vdots & \ddots & \vdots \\
            y_1^{(n-1)}(x) & \cdots & y_n^{(n-1)}(x)
        \end{pmatrix} \ne 0 \]
        Dabei genügt schon \(W(x) \ne 0\) für ein x.
    \end{satz}

    \begin{aufg}
        Berechnen Sie die Wronski-Determinante
        \begin{enumerate}[label=\emph{(\roman*)}]
            \item \(n=2,y_1(x)=x,y_2(x)=x^2\)
            \item \(n=3,y_1(x)=1,y_2(x)=x,y_3(x)=x^2\)
            \item \(n=2,y_1(x)=\sin(x),y_2(x)=17\sin(x)\)
        \end{enumerate}
    \end{aufg}

    \begin{defi}[Fundamentalsystem]
        Sind die Funktionen \(y_1,\dots,y_n\) linear unabhängige Funktionen einer linearen homogenen Differentialgleichung n-ter Ordnung, dann heißt die Menge \(\{y_1,\dots,y_n\}\) ein \underline{Fundamentalsystem} der Differentialgleichung.
    \end{defi}

    \section{Lineare Differentialgleichungen mit konstanten Koeffizienten}

    \begin{defi}[lin.\ homo. Differentialgleichung mit konst. Koeffizienten]
        Die Gleichung \(y^{(n)}(x) + a_{n-1} \cdot y^{(n-1)}(x) + \cdots + a_1 \cdot y'(x) + a_0 \cdot y(x) = 0\) mit \(a_0,\dots,a_{n-1} \in \mathbb{R}\) heißt \underline{lineare homogene Differentialgleichung mit konstanten Koeffizienten}.
    \end{defi}

    \begin{bemerkung}
        Für die Gleichung
        \[y^{(n)}(x) + a_{n-1} \cdot y^{(n-1)}(x) + \cdots + a_1 \cdot y'(x) + a_0 \cdot y(x) = 0\]
        wird ein Ansatz verwendet, der eine e-Funktion enthält.

        \underline{Ansatz:} \(y(x)=e^{\lambda x},\lambda\in\mathbb{R}\)

        Differenzieren und Einsetzen des Ansatzes führt zu: \\ \(e^{\lambda x}\underbrace{(\lambda^n+a_{n-1}\lambda^{n-1}+\cdots+a_1\lambda+a_0)}=0\)
        
        wird 0, falls \(\lambda\) eine Nullstelle des Polynoms \(P(x)=x^n+a_{n-1}x^{n-1}+\cdots+a_1x+a_0\) ist.
    \end{bemerkung}

    \begin{defi}[charakteristisches Polynom]
        Es sei \(y^{(n)}(x) + a_{n-1} \cdot y^{(n-1)}(x) + \cdots + a_1 \cdot y'(x) + a_0 \cdot y(x) = 0\) eine homogene lineare Differentialgleichung mit konstanten Koeffizienten.

        Dann heißt das Polynom \(P(x)=x^n+a_{n-1}x^{n-1}+\cdots+a_1x+a_0\) \\ \underline{charakteristisches Polynom der Differentialgleichung}.
    \end{defi}

    \begin{satz}
        Es sei \(y^{(n)}(x) + a_{n-1} \cdot y^{(n-1)}(x) + \cdots + a_1 \cdot y'(x) + a_0 \cdot y(x) = 0\) eine homogene lineare Differentialgleichung mit konstanten Koeffizienten und \(\lambda \in\mathbb{R}\) eine Nullstelle des charakteristischen Polynoms \(P(x)\), dann ist \(y(x)=e^{\lambda x}\) eine Lösung der Differentialgleichung.
    \end{satz}

    \begin{aufg}
        \begin{enumerate}[label=\emph{(\roman*)}]
            \item Zu lösen ist: \(y''-3y'+2y=0\)
            
            \(P(x)=x^2-3x+2,\lambda_1=1,\lambda_2=2\)

            \(y_1(x)=e^x,y_2(x)=e^{2x}\)

            \item Zu lösen ist \(y'''-y''-2y'=0\)
         \end{enumerate}
    \end{aufg}

    \begin{satz}
        Es sei \(y^{(n)}(x) + a_{n-1} y^{(n-1)}(x) + \cdots + a_1 y'(x) + a_0 y(x) = 0\) eine homogene lineare Differentialgleichung mit konstanten Koeffizienten, deren charakteristisches Polynom \(P(x)\) n verschiedene reelle Nullstellen \(\lambda_1,\dots,\lambda_n\) hat. Dann bilden die Funktionen \(y_1(x)=e^{\lambda_1 x},\dots,y_n(x)=e^{\lambda_n x}\) ein Fundamentalsystem der Differentialgleichung. Jede Lösung der Differentialgleichung hat deshalb die Form \(y(x)=c_1 \cdot e^{\lambda_1 x} + \cdots + c_n \cdot e^{\lambda_n x}\) mit \(c_1,\dots,c_n \in \mathbb{R}\).
    \end{satz}

    \begin{beispiel}
        Zu lösen ist \(y''-2y'+y=0\)

        \(P(x)=x^2-2x+1,\lambda_{1,2}=1\)

        \(y_1(x)=e^x\)

        Ansatz für die zweite Lösung: \(y_2(x)=x \cdot e^x\)
    \end{beispiel}

    \begin{satz}
        Es sei \(y^{(n)}(x) + a_{n-1} y^{(n-1)}(x) + \cdots + a_1 y'(x) + a_0 y(x) = 0\) eine homogene lineare Differentialgleichung mit konstanten Koeffizienten. Ihr charakteristisches Polynom \(P(x)\) habe k reelle Nullstellen \(\lambda_1,\dots,\lambda_k\) mit \(P(x)={(x-\lambda_1)}^{m_1}{(x-\lambda_2)}^{m_2} \dots {(x-\lambda_k)}^{m_k}\), d.h. \(\lambda_j\) ist \(m_j\)-fache Nullstelle von P. Dann bilden die Funktionen
        \begin{align*}
            e^{\lambda_1 x},x \cdot e^{\lambda_1 x},\dots,x^{m_1-1} \cdot e^{\lambda_1 x} \\
            e^{\lambda_2 x},x \cdot e^{\lambda_2 x},\dots,x^{m_2-1} \cdot e^{\lambda_2 x} \\
            \vdots \hspace{1cm} \vdots \hspace{1cm} \\
            e^{\lambda_k x},x \cdot e^{\lambda_k x},\dots,x^{m_k-1} \cdot e^{\lambda_k x}
        \end{align*}
        ein Fundamentalsystem der Differentialgleichung.
    \end{satz}

    \begin{aufg}
        \begin{enumerate}
            \begin{enumerate}[label=\emph{(\roman*)}]
                \item Zu lösen ist \(y''''-3y'''+3y''-y'=0\)
                \\ Lösung:
                \(\lambda_1=0  \lambda_2=1 \) (3-fache NST)
                \\ \(y_1(x)=e^{0x}=1 \) \\
                \(y_2(x)=e^{x} \) \\
                \(y_3(x)=x\cdot e^{x} \) \\
                 \(y_3(x)=x^2\cdot e^{x} \) \\
                \item Zu lösen ist das folgende Anfangswertproblem:
                
                \(y''-4y'+4y=0\) mit \(y(0)=1,y'(0)=1\)
            \end{enumerate}
        \end{enumerate}
    \end{aufg}

     \begin{bemerkung}
         Liegt eine inhomogene Differentialgleichung mit Störfunktion \(b(x)\) vor, so kann in einigen Fällen eine geeignete Ansatzfunktion zur Lösung der Differentialgleichung verwendet werden:

         Liegt zum Beispiel die Störfunktion in der folgenden Form vor

         \(b(x)=a_n x^n + a_{n-1} x^{n-1} + \cdots + a_1 x + a_0\), so kann die folgende Ansatzfunktion verwendet werden:

         \(\alpha_n x^n + \alpha_{n-1} x^{n-1} + \cdots + \alpha_1 x + \alpha_0\)
     \end{bemerkung}

     \begin{beispiel}
        Zu lösen ist \(y'''-3y'-2y=4x^2\)

        Zunächst homogene Differentialgleichung betrachten: \(y'''-3y'-2y=0,P(x)=x^3-3x-2\)

        \(\lambda_1=-1,\lambda_2=2,\lambda_3=-1\)

        Lösung der homogenen Differentialgleichung: \(y(x)=c_1xe^{-x}+c_2e^{-x}+c_3e^{2x}\)

        Ansatz für die Störfunktion: Da \(b(x)=x^2\) ein Polynom 2. Grades ist, wird der Ansatz \(\underbrace{\alpha_2x^2+\alpha_1x+\alpha_0}_{y_P}\) verwendet.

        \({y_P}'''=0,{y_P}''=2\alpha_2,{y_P}'=2\alpha_2x+\alpha_1\)

        \(-6\alpha_2x-3\alpha_1-2\alpha_2x^2-\alpha_1x-2\alpha_0=4x^2\)

        \(x^2\underbrace{(-2\alpha_2)}_4+x\underbrace{(-6\alpha_2-\alpha_1)}_0+\underbrace{(-3\alpha_1-2\alpha_0)}_0=4x^2+0x+0\)

        \(\alpha_2=-2,\alpha_1=6,\alpha_0=-9\)

        Lösung der inhomogenen Differentialgleichung: \(y(x)=c_1xe^{-x}+c_2e^{-x}+c_3e^{2x}-2x^2+6x-9,\ x\in\mathbb{R},c_i\in\mathbb{R}\)
     \end{beispiel}

     \begin{aufg}
         Lösen Sie \(y'''-3y'-2y=100\sin(2x)\)

         \(b(x)=a \cdot \sin(cx) + b \cdot \sin(cx) \to\) Ansatz: \(\alpha \cdot \sin(cx) + \beta \cdot \sin(cx)\) \\
         Lösung: \\
         \( y_p(x)=\alpha\cdot\sin(cx)+\beta\cdot\cos(cx); cx=2 \)
         \\ Koeffizientenvergleich: \\
         \(\alpha=-1, \beta=7\) \\
         \( y(x)=c_1\cdot e^{2x} + c_2\cdot e^{-x} +
          c_3 \cdot e^{-x} - \sin(2x) + 7\cos(2x); x\in \mathbb{R}, c_i\in \mathbb{R} \) \\
          \(y(0)=1 \rightarrow 1=c_1\cdot1+0 \rightarrow c_1=1 \) \\
          \( y'(0)=1 \rightarrow y'(x)=2c_1e^{2x}+c_2e^{2x}+2c_2\cdot x \cdot e^{2x} = 1 \) \\
          \(c_2\cdot xe^{2x}\) \\
          \(c_2(1e^{2x}+x\cdot 2e^{2x})\)
     \end{aufg}

     \section{Differentialgleichungssysteme (DGLS)}

     \begin{defi}[Differentialgleichungssysteme]
         Ein System von m Gleichungen, dass die unbekannten Funktionen \(y_1(x),y_2(x),\dots,y_m(x)\) sowie deren Ableitungen \\
         \({y_1}'(x),{y_1}''(x),\dots,{y_1}^{(n_1)}(x),\dots,{y_m}'(x),{y_m}''(x),\dots,{y_m}^{(n_m)}(x)\) enthält, heißt \\ \underline{Differentialgleichungssystem}.
     \end{defi}

     \begin{beispiel}
         Zu lösen ist das Differentialgleichungssystem
         \begin{align*}
             {y_1}' &= -2y_1+8y_2\ (*) \\
             {y_2}' &= -4y_1+6y_2+10x^2+16x-8\ (**) \\
         \end{align*}
         Eliminationsmethode:

         \((*)\) ableiten: \({y_1}'' = -2{y_1}'+8{y_2}'\)

         \((**)\) einsetzen: \({y_1}'' = -2{y_1}'-32y_1+48y_2+80x^2+128x-64\) \\
         \( y_1'=-2y_1+8y_2 (*) \) \\
         \(y_2')-4y_1+6y_2+10x^2+16x-8 (**) \) \\
         \( y_1''=-2y_1+8(-4y_1+6y_2+10x^2+16x-8) \) \\
         \( =-2y_1'-32y_1+48y_2+80x^2+128x-64  \overline{(*)}' \) \\
         \( (*) \) umstellen nach \( y_2 \) \\ 
         \( y_2=\frac{1}{8}y_1'+\frac{1}{4}y_1 \) \\
         Einsetzen in \( \overline{(*)}' \) \\
         \( \rightarrow y_1''=-2y_1'=-32y_1+48(\frac{1}{8}y_1'+\frac{1}{4}y_1) + 80x^2+128x-64 \) \\
         \(y_1'(-2+6)+y_1(-32+12)+89x^2+128x-64 \) \\
         \(\rightarrow y_1'''-4y_1'+20y_1=80x^2+128x-64 (***) \) \\
         \(\rightarrow \) homogene DGL \(\rightarrow \) charakteristisches Polynom: \(P(x)=x^2-4x+20 \) \\
         \(a=0; b=-4;c=20\) \\
         \(\lambda_{1,2}=2\pm4i \) \\
         \(y_1=\cos(4x)\cdot e^{2x} \) \\
         \(y_2=\sin(4x)\cdot e^{2x} \) (Eulersche Form)\\
         \(e^{\lambda x}=e^{(2+4i)x}=e^{2x}(\cos(4x)+i\sin(4x)) \) \\
         Ansatz für die inhomogene Lösung: \\
         \(y_p=\alpha_2x^2+\alpha_1x+\alpha_0 \) \\
         \(y_p'=2\alpha_2x+\alpha_1, y_p''=2\alpha_2, y_p'''(x)=0 \) \\
         Einsetzen in (***) für \(y_1, y_1', y_1''\) \\
         \( 2\alpha_2-4(2\alpha_2x+\alpha_1)+20(\alpha_2x^2+\alpha_1x+\alpha_0) = 80x^2+128x-64 \) \\
         Sortieren: \(x^2(20\alpha_2)+x(-8\alpha_2+20\alpha_1)+(2\alpha_2-4\alpha_1+20\alpha_0)\)
         \\ Koeffizientenvergleich \(20\alpha_2 = 80 \) usw. \\
         \(\alpha_0=-2 \)\\
         \(\alpha_1=8 \)\\
         \(\alpha_2=4 \)\\
         \( y_1(x)=\underbrace{c_1\cdot\cos(4x)e^{2x}+c_2\cdot\sin(4x)e^{2x}}_{\text{Homogone LSG}}+\underbrace{4x^2+8x-2}_{\text{Partikuläre LSG}}\)  (****) \\ \(x\in\mathbb{R}\) \\ \(c_1,c_2\in\mathbb{R}\) \\
         \(y_2=\frac{1}{8}y_1'+\frac{1}{4}y_1\)\\
         \(y_1'(x)=-4c_1\cdot\sin(4x)e^{2x}+2c_1\cdot\cos(4x)e^{2x}+4c_2\cdot\cos(4x)e^{2x}+2c_2\cdot\sin(4x)e^{2x}+8x+8 \) \\
         \(=e^{2x}\cdot\cos(4x)(
         \frac{1}{2}c_1+\frac{1}{2}c_2
         )+e^{2x}\cdot\sin(4x)(
         -\frac{1}{2}c_1+\frac{1}{2}c_2
         )+x^2+3x+\frac{1}{2} \) \\
         Vektorschreibweise: \\
         \(c_1=2D_1\) \\
         \(c_2=2D_2\) \\
         \( \vec{y}(x)=\begin{pmatrix}
         y_1(x) \\ y_2(x)
         \end{pmatrix}=\begin{pmatrix}2D_1 \\ D_1+D_2 \end{pmatrix}\cdot e^{2x}\cdot\cos(4x) \)\\
         
         
     \end{beispiel}
\end{document}          
