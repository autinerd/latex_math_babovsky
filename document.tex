% !TeX encoding = UTF-8
\documentclass[12pt,a4paper]{scrreprt}
\usepackage[utf8]{inputenc}
\newtheorem{defi}{Definition}[section]
\newtheorem{bemerkung}[defi]{Bemerkung}
\newtheorem{beispiel}[defi]{Beispiel}
\newtheorem{satz}[defi]{Satz}
\newtheorem{aufg}[defi]{Aufgabe}
\usepackage{amsmath}
\usepackage{amsfonts}
\usepackage{amssymb}
\usepackage{enumitem}


\begin{document}
	
	\chapter{Grundlagen der Differentialrechnung reeller Funktionen mit mehreren Variablen}
	
	\section{Partielle Ableitungen}
	
	\begin{defi}[reelle Funktionen mit n Variablen]
		Eine Funktion $y = f(x_1, \dots, x_n)$ mit $\left(x_1, \dots, x_n\right) \mathbb{D} \subset \mathbb{R}^n$ und $y \in \mathbb{R}$ heißt reelle Funktion mit mehreren Variablen.
		$\mathbb{D}$ beschreibt den Definitionsbereich und wir schreiben $f:\mathbb{D} \to \mathbb{R}$
	\end{defi}

\begin{bemerkung}
	Definition für Differenzierbarkeit im eindimensionalen Fall:
	Es sei $f : \mathbb{D} \to \mathbb{R}$ eine Funktion mit $\mathbb{D} \subset \mathbb{R}$ und $x_0 \in \mathbb{R}$
	f ist differenzierbar in $x_0$, falls der Grenzwert $\lim_{x\to x_0} {\frac{f\left(x\right) - f\left(x_0\right)}{x-x_0}} \in \mathbb{R}$ existiert.
	In diesem Fall heißt $f'\left(x_0\right) = \lim_{x\to x_0}{\frac{f\left(x\right) - f\left(x_0\right)}{x-x_0}}$ die Ableitung von f in $x_0$
	Man bezeichnet $f'\left(x_0\right)$ als den Differentialquotienten von f im Punkt $x_0$.
	
	$f'\left(x_0\right) = \lim_{\Delta x\to0}{\frac{f\left(x_0 + \Delta x\right) - f\left(x_0\right)}{\Delta x}}$
	
\end{bemerkung}

\begin{defi}
	$P=(p_1,\ldots,p_n)$ und $Q=(q_1,\ldots,q_n)$ bezeichnen zwei Punkte im n-dimensionalen Raum $\mathbb{R}^n$
	
	$\left|P-Q\right|=\sqrt{\left(p_1-q_1\right)^2+\cdots+\left(p_n+q_n\right)^2}$ heißt Abstand der Punkte P und Q.
	
	Die Delta-Umgebung des Punktes P ist eine Teilmenge des $\mathbb{R}^n$ mit der Eigenschaft $U_\delta\left(P\right)=\left\{Q\in\mathbb{R}^n:\left|Q-P\right|<\delta\right\}$.
	
\end{defi}
	\begin{defi}
		Der Punkt P heißt innerer Punkt der Menge M $(M \subset R^n)$, wenn eine Umgebung des Punktes P existiert, für die $U_\delta \subset M$ gilt.
	\end{defi}

\begin{bemerkung}
	Eine Menge heißt \underline{offene Menge}, wenn dir nur aus inneren Punkten besteht.
\end{bemerkung}

\begin{defi}
	Wenn \((x_{0n},\dots,x_{0n})\) ein innerer Punkt der Menge D ist und wenn der Grenzwert existiert, dann heißt die Funktion \(f(x_1,\dots,x_n)\) an der Stelle \((x_{01},\dots,x_{0n})\in D\) nach \(x_i\) partiell differenzierbar. Den Grenzwert bezeichnet man als partielle Ableitung der Funktion f nach \(x_i\) an der Stelle \((x_{01},\dots,x_{0n}) \in D\)
	
	Die Funktion f heißt in \((x_{01},\dots,x_{0n}) \in D\) partiell differenzierbar, wenn die partiellen Ableitungen nach allen Komponenten \(x_j (j=1,\dots,n)\) existieren.
	
	Die Funktion heißt in $\mathbb{D}$ partiell differenzierbar, wenn $f$ in allen inneren Punkten aus $\mathbb{D}$ partiell differenzierbar ist.
\end{defi}

\begin{bemerkung}
	Die partielle Ableitung der Funktion \(f(x_1,\dots,x_n)\) nach der Komponente \(x_j\) kann wie folgt bezeichnet werden:
	
	\(f_{x_j}(x_1,\dots,x_n)\) oder \(\frac{\partial f(x_1,\dots,x_n)}{\partial x_j}\)
\end{bemerkung}
\begin{beispiel}
	Gesucht sind die partiellen Ableitungen der Funktion
	\(f(x_1,x_2,x_3)=x_3*\sin(x_1^2+x_2)+e^{2x_3}\)
	
	\[\frac{\partial f}{\partial x_1}(x_1,x_2,x_3)=x_3\cdot\cos(x_1^2+x_2)\cdot2x_1\]
	
	\[\frac{\partial f}{\partial x_2}(x_1,x_2,x_3)=x_3\cdot\cos(x_1^2+x_2)\]
		
	\[\frac{\partial f}{\partial x_3}(x_1,x_2,x_3)=\sin(x_1^2+x_2)\cdot2e^{2x_3}\]
\end{beispiel}

\begin{bemerkung}
	Die Tangentialebene an die Funktion $f(x_1,x_2)$ berührt die Funktion $f(x_1,x_2)$ im Punkt $\bar{P}=(\bar{x_1},\bar{x_2},f(\bar{x_1},\bar{x_2}))$ und enthält alle Tangenten an die Funktion $f(x_1,x_2)$ im Punkt $\bar{P}$.
\end{bemerkung}

\begin{satz}
Es seien $\mathbb{D} \subset \mathbb{R}^2, f:\mathbb{D}->\mathbb{R}$ eine partiell differenzierbare Funktion und $(x_{01},x_{02})\in\mathbb{D}$. Dann lautet die Gleichung der Tangentialebene für den Punkt $(x_{01},x_{02})$:

\[x_3=f(x_{01},x_{02})+\frac{\partial f}{\partial x_1}(x_{01},x_{02})(x_1-x_{01})+\frac{\partial f}{\partial x_2}(x_{01},x_{02})(x_2-x_{02})\]

\[=f(x_{01},x_{02})+\begin{pmatrix} \frac{\partial f}{\partial x_1} (x_{01},x_{02}) & \frac{\partial f}{\partial x_2}(x_{01},x_{02}) \end{pmatrix} \cdot \begin{pmatrix} (x_1-x_{01})\\ (x_2-x_{02}) \end{pmatrix}\]
\end{satz}

\begin{beispiel}
	Gegeben sei $f:\mathbb{R}^2->\mathbb{R}$ durch $f(x_1,x_2)=\sin(x_1\cdot x_2^2)$

	$\frac{\partial f}{\partial x_1}(x_1,x_2)=\cos(x_1\cdot x_2^2)\cdot x_2^2$
	
	$\frac{\partial f}{\partial x_2}(x_1,x_2)=\cos(x_1\cdot x_2^2)\cdot 2x_2 \cdot x_1$
	
	$\to$ Tangentialebene: $x_3 = f(x_{01},x_{02}) + \cos(x_1\cdot x_2^2)\cdot x_2^2 \cdot (x_1 - x_{01}) + \cos(x_1\cdot x_2^2)\cdot 2x_2 \cdot x_1 \cdot (x_2 - x_{02})$
	
	$(x_{01},x_{02}) = (0,0)$
	
	$x_3 = 0 + \begin{pmatrix}0 & 0\end{pmatrix} \begin{pmatrix}x_1 - 0 \\ x_2 - 0\end{pmatrix} = 0$
	
\end{beispiel}

\begin{beispiel}
	
	$(x_{01},x_{02}) = (\sqrt[3]{\pi},\sqrt[3]{\pi})$
	
	$f(\sqrt[3]{\pi},\sqrt[3]{\pi})=\sin(\sqrt[3]{\pi}\cdot \sqrt[3]{\pi}^2) = 0$
	
	$\frac{\partial f}{\partial x_1}(\sqrt[3]{\pi},\sqrt[3]{\pi})=\cos(\sqrt[3]{\pi}\cdot \sqrt[3]{\pi}^2)\cdot \sqrt[3]{\pi}^2 = -\pi^{\frac{2}{3}}$
	
	$\frac{\partial f}{\partial x_2}(\sqrt[3]{\pi},\sqrt[3]{\pi})=\cos(\sqrt[3]{\pi}\cdot \sqrt[3]{\pi}^2)\cdot 2\sqrt[3]{\pi} \cdot \sqrt[3]{\pi} = -2\pi^{\frac{2}{3}}$
	
	$x_3 = 0 + \begin{pmatrix}-\pi^{\frac{2}{3}} & -2\pi^{\frac{2}{3}}\end{pmatrix} \begin{pmatrix}x_1 - \sqrt[3]{\pi} \\ x_2 - \sqrt[3]{\pi}\end{pmatrix} = -\pi^{\frac{2}{3}} \cdot (x_1 - \sqrt[3]{\pi}) + -2\pi^{\frac{2}{3}} \cdot (x_2 - \sqrt[3]{\pi})$
	
	$x_3 = -\pi^{\frac{2}{3}}x_1 - 2\pi^{\frac{2}{3}}x_2 + 3\pi$
\end{beispiel}

\begin{beispiel}
	Gegeben sei $f:\mathbb{R}^2\to\mathbb{R}$ durch $f(x_1,x_2)=\left| x_1 \right| + x_2$
	$\frac{\partial f}{\partial x_2}(x_1,x_2)=1$
	$\left| x_1 \right|$ ist nur für $x_1 \neq 0$ differenzierbar, d.h. f ist nicht auf dem gesamten Definitionsbereich partiell differenzierbar.
\end{beispiel}

\begin{defi}[Gradient]
	Es sei $\mathbb{D} \subset \mathbb{R}^n$ und $f:\mathbb{D} \to \mathbb{R}$ partiell differenzierbar. Dann heißt der Vektor $\mathrm{grad} f(x) = \begin{pmatrix}
	\frac{\partial f}{\partial x_1}(x) \\
	\frac{\partial f}{\partial x_2}(x) \\
	\vdots \\
	\frac{\partial f}{\partial x_n}(x) \\
	\end{pmatrix}$ der Gradient von $f$ im Punkt $x\in\mathbb{D}=(x_1,\dots,x_n)$.
\end{defi}

\begin{bemerkung}
	Anstelle von $\mathrm{grad} f(x)$ wird auch häufig $\nabla f(x)$ geschrieben.
\end{bemerkung}

\begin{beispiel}
	Berechnen Sie den Gradienten für:
	\begin{enumerate}
		\item $f(x_1,x_2) = x_1^2 + x_2^2$
		\item $g(x_1,x_2,x_3) = 2 \cdot \sin(x_1 x_2) + x_1 x_2 x_3$
	\end{enumerate}

	zu 1. $\nabla f(x_1,x_2) = \begin{pmatrix}2x_1\\2x_2\end{pmatrix}$
	
	zu 2. $\nabla g(x_1,x_2,x_3) = \begin{pmatrix}2x_2 \cos(x_1 x_2)+x_2 x_3 \\ 2x_1 \cos(x_1 x_2)+x_1 x_3 \\ x_1 x_2\end{pmatrix}$
\end{beispiel}

\begin{beispiel}
	Partielle Differenzierbarkeit impliziert nicht Stetigkeit.
	
	Betrachtet wird die Funktion $f:\mathbb{R}^2\to\mathbb{R}$ mit
	
	\[f(x_1,x_2) = \begin{cases}
	\frac{x_1\cdot x_2}{x_1^2 + x_2^2} & (x_1,x_2) \ne (0,0) \\
	0 & (x_1,x_2) = (0,0)
	\end{cases}\]
	
	Im Punkt $(0,0)$ existieren die partiellen Ableitungen:
	
	\[\frac{\partial f}{\partial x_1}(0,0) = \lim_{\Delta x\to0}{\frac{f(0+\Delta x,0)-f(0,0)}{\Delta x}} = 0\]
	
	\[\frac{\partial f}{\partial x_2}(0,0) = \lim_{\Delta x\to0}{\frac{f(0,0+\Delta x)-f(0,0)}{\Delta x}} = 0\]
	
	Aber $f$ ist in $(0,0)$ nicht stetig:
	
	Es gilt $f(x_1,0) = 0; x_1 \in \mathbb{R}$, $f(0,x_2) = 0; x_2 \in \mathbb{R}$
	
	Für $x := x_1 = x_2$:
	
	$f(x,x)=\begin{cases}\frac{x^2}{2x^2}=\frac{1}{2} & (x,x) \ne (0,0) \\ 0 & (x,x) = (0,0)\end{cases}$
\end{beispiel}

\begin{defi}[Stetigkeit]
	Die Funktion $y=f(x_1,\dots,x_n)$, $(x_1,\dots,x_n)\in\mathbb{D}$ ist an der Stelle $\bar{P}=(\bar{x_1},\dots,\bar{x_n}) \in \mathbb{D}$ stetig, wenn für den Funktionsgrenzwert
	
	\[\lim_{P\to\bar{P}}{f(x_1,\dots,x_n)} = f(\bar{x_1},\dots,\bar{x_n})\] gilt.
\end{defi}

Aufgaben:

\begin{enumerate}
	\item Berechnen Sie alle ersten und zweiten partiellen Ableitungen der Funktion $f(x_1,x_2)=e^{-2.5x_1^2-(x_2-1)^2}, (x_1,x_2) \in \mathbb{R}$
	\item Berechnen die den Gradienten der Funktion $f(x_1,x_2,x_3)=\frac{(x_1-1)\cdot\ln(x_1+1)}{x_2^2+x_3^2+1}$ an der Stelle $(0,0,0)$
	\item Berechnen Sie die Tangentialebene an die Funktion  $f(x_1,x_2)=e^{-2.5x_1^2-(x_2-1)^2}, (x_1,x_2) \in \mathbb{R}$ im Punkt $(0,\frac{3}{2},e^{-\frac{1}{4}})$
\end{enumerate}

%%% insert missing parts

\section{Totale Differenzierbarkeit}

\begin{defi}[Betrag eines Vektors]
	Der Betrag eines Vektors $x=(x_1,\dots,x_n)\in\mathbb{R}^n$ ist definiert als
\end{defi}

\section{Extremwerte}

\begin{defi}[Lokales Extremum]
	Es seien $U\subset\mathbb{R}^n$ eine offene Menge und $f:U\to\mathbb{R}$ eine Abbildung.
	\begin{enumerate}[label=(\roman*)]
		\item Ein Punkt $x_0\in U$ heißt \underline{lokales Maximum} von $f$, falls $f$ in der Nähe von $x_0$ nicht größer wird als bei $x_0$, das heißt:
		$f(x)\le f(x_0)$ für alle $x$ in der Nähe von $x_0$.
		\item Ein Punkt $x_0\in U$ heißt \underline{lokales Minimum} von $f$, falls $f$ in der Nähe von $x_0$ nicht kleiner wird als bei $x_0$, das heißt:
		$f(x)\ge f(x_0)$ für alle $x$ in der Nähe von $x_0$.
	\end{enumerate}
	Ein \underline{lokales Extremum} ist ein lokales Minimum oder ein lokales Maximum.
\end{defi}

\begin{bemerkung}
	Wie im eindimensionalen Fall liefert die Differentialrechnung nur Informationen über lokale und nicht über globale Extrema.
\end{bemerkung}

\begin{bemerkung}
	Es sei $f:(a,b)\to\mathbb{R}$ differenzierbar in $x_0\in (a,b)$ mit einem lokalen Extremum in $x_0$. Dann ist $f'(x_0)=0$.
\end{bemerkung}

\begin{satz}
	Es sei $U \subset \mathbb{R}^n$ eine offene Menge und $f:U\to\mathbb{R}$ differenzierbar. Besitzt f in $x_0\in U$ ein lokales Extremum, so gilt:
	\[\nabla f(x_0)=0,\]
	d.h. $\frac{\partial f}{\partial x_1}(x_0)=\dots=\frac{\partial f}{\partial x_n}(x_0)=0$
\end{satz}

\begin{beispiel}
	$f:\mathbb{R}^2\to\mathbb{R}$\hspace{1cm}$f(x_1,x_2)=1-x_1^2-x_2^2$
	
	\[\frac{\partial f}{\partial x_1}(x_1,x_2)=-2x_1\hspace{1cm}\frac{\partial f}{\partial x_2}(x_1,x_2)=-2x_2\]
	
	$\to x_1=x_2=0$
	
	D.h. falls es ein Extremum gibt, dann liegt es bei (0,0).
\end{beispiel}

\begin{aufg}
	\[f:\mathbb{R}^2\to\mathbb{R}\hspace{1cm}f(x_1,x_2)=\sin(x_1)\cdot\sin(x_2)\]
\end{aufg}

\begin{bemerkung}
	Es sei $f:(a,b)\to\mathbb{R}$ zweimal differenzierbar und $f'(x_0)=0$.
	\begin{enumerate}[label=(\roman*)]
		\item Ist $f''(x_0)>0$, so hat f in $x_0$ ein lokales Minimum.
		\item Ist $f''(x_0)<0$, so hat f in $x_0$ ein lokales Maximum.
	\end{enumerate}
\end{bemerkung}

\begin{defi}
	Eine Matrix $A=(a_{ij}), i,j=1,\dots,n \in \mathbb{R}^{n\times n}$ heißt \underline{positiv definit}, falls gilt:
	$a_{11}>0,\det\begin{pmatrix}
	a_{11} & a_{12} \\ a_{21} & a_{22}
	\end{pmatrix} > 0, \dots, \det\begin{pmatrix}
	a_{11} & \cdots & a_{1n} \\
	a_{21} &  & a_{2n} \\
	\vdots & & \vdots \\
	a_{n1} & \cdots & a_{nn}
	\end{pmatrix} > 0$, also $\det\begin{pmatrix}
	a_{11} & \cdots & a_{1k} \\
	a_{21} &  & a_{2k} \\
	\vdots & & \vdots \\
	a_{k1} & \cdots & a_{kk}
	\end{pmatrix} > 0$ für alle $k=1,\dots,n$.
	
	A heißt \underline{negativ definit}, falls -A positiv definit ist.
\end{defi}

\begin{beispiel}
	Es sei $A=\begin{pmatrix}
	1 & 1 \\ 1 & 4
	\end{pmatrix} \in \mathbb{R}^{2\times 2}$
	
	$\underline{k=1}:1>0$
	
	$\underline{k=2}: \det\begin{pmatrix}
	1 & 1 \\ 1 & 4
	\end{pmatrix} = 1\cdot4-1\cdot1=3>0$
	
	$\to$ A ist positiv definit.
\end{beispiel}

\begin{aufg}
	$B=\begin{pmatrix}
	-6 & 2 \\ 2 & -1
	\end{pmatrix} \in \mathbb{R}^{2\times 2}$
	
	$-6<0 \to$ B nicht positiv definit
	
	$-B=\begin{pmatrix}
	6 & -2 \\ -2 & 1
	\end{pmatrix}$
	
	$\underline{k=1}:6>0$
	
	$\underline{k=2}: \det\begin{pmatrix}
	6 & -2 \\ -2 & 1
	\end{pmatrix} = 6\cdot1 - (-2)\cdot(-2)=2>0$
	
	$\to$ B ist negativ definit.
\end{aufg}

\begin{aufg}
	Prüfen Sie $C=\begin{pmatrix}
	1 & 1 & 0 \\ 1 & 3 & 1 \\ 0 & 1 & 2
	\end{pmatrix} \in \mathbb{R}^{3\times3}$ auf positive Definitheit.
\end{aufg}

\begin{bemerkung}
	$D=\begin{pmatrix}
	0 & 1 \\ 1 & 4
	\end{pmatrix} \in \mathbb{D}^{2\times2}$ ist weder positiv noch negativ definit: $d_{11}\not>0\to$ nicht positiv definit
	
	$-d_{11}\not>0\to$ nicht negativ definit
\end{bemerkung}

\begin{defi}
	Es seien $U\subset\mathbb{R}^n$ eine offene Menge, $f:U\to\mathbb{R}$ eine zweimal stetig partiell differenzierbare Funktion und $x_0\in U$. Unter der \underline{Hesse-Matrix} von f in $x_0$ versteht man die Matrix: \[H_f(x_0)=\begin{pmatrix}
	\frac{\partial^2 f}{\partial x_1^2}(x_0) & \cdots & \frac{\partial^2 f}{\partial x_1 \partial x_n}(x_0) \\
	\frac{\partial^2 f}{\partial x_2 \partial x_1}(x_0) & \cdots & \frac{\partial^2 f}{\partial x_2 \partial x_n}(x_0) \\
	\vdots & \ddots & \vdots \\
	\frac{\partial^2 f}{\partial x_n \partial x_1}(x_0) & \cdots & \frac{\partial^2 f}{\partial x_n^2}(x_0) \\
	\end{pmatrix}\]
\end{defi}

\begin{beispiel}
	Gegeben sei $f:\mathbb{R}^2\to\mathbb{R}$ durch $f(x_1,x_2)=x_1^2+x_2^2$. Gesucht ist die Hesse-Matrix.
	
	\[H_f(x_1,x_2)=\begin{pmatrix}
	\frac{\partial^2 f}{\partial x_1^2}(x_1,x_2)  & \frac{\partial^2 f}{\partial x_1 \partial x_2}(x_1,x_2) \\
	\frac{\partial^2 f}{\partial x_2 \partial x_1}(x_1,x_2) & \frac{\partial^2 f}{\partial x_2^2}(x_1,x_2)
	\end{pmatrix}=\begin{pmatrix}
	2 & 0 \\ 0 & 2
	\end{pmatrix}\]
\end{beispiel}

\begin{aufg}
	Gegeben sei $f:\mathbb{R}^3\to\mathbb{R}$ durch
	\[f(x_1,x_2,x_3)=x_1^3 \cdot x_2^3 \cdot \sin(x_3)\]
	Berechnen Sie die Hesse-Matrix $H_f(0,0,0)$ \& $H_f(1,1,0)$.
\end{aufg}

\begin{satz}
	Es sei $U\subset\mathbb{R}^n$ eine offene Menge, $f:U\to\mathbb{R}$ zweimal stetig differenzierbar und $x_0\in U$ ein Punkt mit $\nabla f(x_0)=0$.
	\begin{enumerate}[label=(\roman*)]
		\item Ist $H_f(x_0)$ positiv definit, so hat f in $x_0$ ein lokales Minimum.
		\item Ist $H_f(x_0)$ negativ definit, so hat f in $x_0$ ein lokales Maximum.
		\item Ist $\underline{U\subset\mathbb{R}^2}$ und gilt $\det H_f(x_0) < 0$, so liegt kein Extremwert vor.
	\end{enumerate}
\end{satz}

\begin{beispiel}
	Gegeben sei $f:\mathbb{R}^2\to\mathbb{R}$ durch $f(x_1,x_2)=1+x_1^2+x_2^2$
	
	$\nabla f(x_1,x_2)=\begin{pmatrix}
	2x_1 \\ 2x_2
	\end{pmatrix} := 0 \to x_1=x_2=0$
	
	$\to (0,0)$ könnte ein Extremum sein
	
	Überprüfen durch Hesse-Matrix
	
	$H_f(x_1,x_2)=\begin{pmatrix}
	2 & 0 \\ 0 & 2
	\end{pmatrix}$
	
	$2>0, \det\begin{pmatrix}
	2 & 0 \\ 0 & 2
	\end{pmatrix}=4>0$
	
	$\to H_f(0,0)$ positiv definit $\to$ in $(0,0)$ liegt ein lokales Minimum vor.
\end{beispiel}

\begin{aufg}
	Gegeben sei $f:\mathbb{R}^2\to\mathbb{R}$ durch $f(x_1,x_2)=\sin(x_1)\cdot\sin(x_2)$
	
	Untersuchen Sie die Funktion auf lokale Extrema.
	
	Kandidaten: $(0,0),(\frac{\pi}{2},\frac{\pi}{2})$
\end{aufg}

\begin{aufg}
	Gegeben sei die Funktion $f:\mathbb{R}^2\to\mathbb{R}$ durch $f(x_1,x_2)=\cos(x_1)+\cos(x_2)$
	
	Gradient: $\nabla f(x_1,x_2)=\begin{pmatrix}
	-\sin x_1 \\ -\sin x_2
	\end{pmatrix} := \begin{pmatrix}
	0 \\ 0
	\end{pmatrix} \to -\sin x_1 = -\sin x_2 = 0$
	
	$(k_1\pi,k_2\pi), k_1,k_2\in\mathbb{Z}$
	
	Hesse-Matrix
	
	$H_f(x_1,x_2)=\begin{pmatrix}
	-\cos x_1 & 0 \\ 0 & -\cos x_2
	\end{pmatrix}$
	
	$H_f(k_1\pi,k_2\pi)=\begin{pmatrix}
	-(-1)^{k_1} & 0 \\ 0 & -(-1)^{k_2} 
	\end{pmatrix}$
	
	\vspace{1cm}
	\def\arraystretch{1.25}	
	\begin{tabular}{|c|c|c|} \hline
	& $k_1$ gerade & $k_1$ ungerade \\ \hline
	$k_2$ gerade & lokales Maximum & kein Extremwert \\ \hline
	$k_2$ ungerade & kein Extremwert & lokales Minimum \\ \hline
	\end{tabular}
\end{aufg}

\begin{beispiel}[Nebenbedingungen]
	Gegeben sei 12m langer Draht, aus dem die Kanten eines Quaders von möglichst großem Volumen hergestellt werden sollen. Gesucht sind die Kantenlängen $x_1,x_2,x_3$ des optimalen Quaders.
	
	$4x_1+4x_2+4x_3=4(x_1+x_2+x_3)=12$
	
	$x_1+x_2+x_3=3$
	
	$V=x_1\cdot x_2 \cdot x_3$
	
	$x_1,x_2,x_3 > 0$
	
	$V=x_1 x_2 (3-x_1-x_2)$
	
	$V=3x_1x_2-x_1^2x_2-x_1x_2^2$
	
	$\mathbb{D}={(x_1,x_2)\in\mathbb{R}^2:x_1>0,x_2>0,x_1+x_2<3}$, $\mathbb{D}$ ist eine offene Menge
	
	$\nabla V(x_1,x_2)=\begin{pmatrix}
	3x_2-2x_1x_2-x_2^2 \\
	3x_1-x_1^2-2x_1x_2
	\end{pmatrix} := \begin{pmatrix}
	0 \\ 0
	\end{pmatrix}$
	
	$\to x_1=1, x_2=1, x_3=1$
	
	Berechnen Sie die Hesse-Matrix
\end{beispiel}

\begin{aufg}
	Papula S.332 zu Abschnitt 2 -> Aufg. 24
	
	\begin{enumerate}
		\item $f(x_1,x_2)=x_1^2(1-x_2)-x_2^3+12x_2+13$
		\item $f(x_1,x_2)=(x_1-1)^2(1-x_2)-x_2^3+12x_2+3$
		\item $f(x_1,x_2)=4(x_1^2-25)(x_2-2)+5x_2^2+12x_2$
	\end{enumerate}
\end{aufg}



\end{document}          
