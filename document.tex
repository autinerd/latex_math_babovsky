% !TeX encoding = UTF-8
\documentclass[12pt,a4paper]{scrreprt}
\usepackage[utf8]{inputenc}
\newtheorem{defi}{Definition}[section]
\newtheorem{bemerkung}[defi]{Bemerkung}
\newtheorem{beispiel}[defi]{Beispiel}
\newtheorem{satz}[defi]{Satz}
\newtheorem{aufg}[defi]{Aufgabe}
\usepackage{amsmath}
\usepackage{amsfonts}
\usepackage{amssymb}
\usepackage{enumitem}
\usepackage{pgfplots}
\usepackage[autostyle=true,german=quotes]{csquotes}
\pgfplotsset{compat=1.16}


\begin{document}
	
	\chapter{Grundlagen der Differentialrechnung reeller Funktionen mit mehreren Variablen}
	
	\section{Partielle Ableitungen}
	
	\begin{defi}[reelle Funktionen mit n Variablen]
		Eine Funktion $y = f(x_1, \dots, x_n)$ mit $\left(x_1, \dots, x_n\right) \mathbb{D} \subset \mathbb{R}^n$ und $y \in \mathbb{R}$ heißt reelle Funktion mit mehreren Variablen.
		$\mathbb{D}$ beschreibt den Definitionsbereich und wir schreiben $f:\mathbb{D} \to \mathbb{R}$
	\end{defi}

\begin{bemerkung}
	Definition für Differenzierbarkeit im eindimensionalen Fall:
	Es sei $f : \mathbb{D} \to \mathbb{R}$ eine Funktion mit $\mathbb{D} \subset \mathbb{R}$ und $x_0 \in \mathbb{R}$
	f ist differenzierbar in $x_0$, falls der Grenzwert $\lim_{x\to x_0} {\frac{f\left(x\right) - f\left(x_0\right)}{x-x_0}} \in \mathbb{R}$ existiert.
	In diesem Fall heißt $f'\left(x_0\right) = \lim_{x\to x_0}{\frac{f\left(x\right) - f\left(x_0\right)}{x-x_0}}$ die Ableitung von f in $x_0$
	Man bezeichnet $f'\left(x_0\right)$ als den Differentialquotienten von f im Punkt $x_0$.
	
	$f'\left(x_0\right) = \lim_{\Delta x\to0}{\frac{f\left(x_0 + \Delta x\right) - f\left(x_0\right)}{\Delta x}}$
	
\end{bemerkung}

\begin{defi}
	$P=(p_1,\ldots,p_n)$ und $Q=(q_1,\ldots,q_n)$ bezeichnen zwei Punkte im n-dimensionalen Raum $\mathbb{R}^n$
	
	$\left|P-Q\right|=\sqrt{\left(p_1-q_1\right)^2 + \cdots + \left( p_n+q_n \right)^2}$ heißt Abstand der Punkte P und Q.
	
	Die Delta-Umgebung des Punktes P ist eine Teilmenge des $\mathbb{R}^n$ mit der Eigenschaft $U_\delta\left(P\right)=\left\{Q\in\mathbb{R}^n:\left|Q-P\right|<\delta\right\}$.
	
\end{defi}
	\begin{defi}
		Der Punkt P heißt innerer Punkt der Menge M $(M \subset R^n)$, wenn eine Umgebung des Punktes P existiert, für die $U_\delta \subset M$ gilt.
	\end{defi}

\begin{bemerkung}
	Eine Menge heißt \underline{offene Menge}, wenn dir nur aus inneren Punkten besteht.
\end{bemerkung}

\begin{defi}
	Wenn \((x_{0n},\dots,x_{0n})\) ein innerer Punkt der Menge D ist und wenn der Grenzwert existiert, dann heißt die Funktion \(f(x_1,\dots,x_n)\) an der Stelle \((x_{01},\dots,x_{0n})\in D\) nach \(x_i\) partiell differenzierbar. Den Grenzwert bezeichnet man als partielle Ableitung der Funktion f nach \(x_i\) an der Stelle \((x_{01},\dots,x_{0n}) \in D\)
	
	Die Funktion f heißt in \((x_{01},\dots,x_{0n}) \in D\) partiell differenzierbar, wenn die partiellen Ableitungen nach allen Komponenten \(x_j (j=1,\dots,n)\) existieren.
	
	Die Funktion heißt in $\mathbb{D}$ partiell differenzierbar, wenn $f$ in allen inneren Punkten aus $\mathbb{D}$ partiell differenzierbar ist.
\end{defi}

\begin{bemerkung}
	Die partielle Ableitung der Funktion \(f(x_1,\dots,x_n)\) nach der Komponente \(x_j\) kann wie folgt bezeichnet werden:
	
	\(f_{x_j}(x_1,\dots,x_n)\) oder \(\frac{\partial f(x_1,\dots,x_n)}{\partial x_j}\)
\end{bemerkung}
\begin{beispiel}
	Gesucht sind die partiellen Ableitungen der Funktion
	\(f(x_1,x_2,x_3)=x_3*\sin(x_1^2+x_2)+e^{2x_3}\)
	
	\[\frac{\partial f}{\partial x_1}(x_1,x_2,x_3)=x_3\cdot\cos(x_1^2+x_2)\cdot2x_1\]
	
	\[\frac{\partial f}{\partial x_2}(x_1,x_2,x_3)=x_3\cdot\cos(x_1^2+x_2)\]
		
	\[\frac{\partial f}{\partial x_3}(x_1,x_2,x_3)=\sin(x_1^2+x_2)\cdot2e^{2x_3}\]
\end{beispiel}

\begin{bemerkung}
	Die Tangentialebene an die Funktion $f(x_1,x_2)$ berührt die Funktion $f(x_1,x_2)$ im Punkt $\bar{P}=(\bar{x_1},\bar{x_2},f(\bar{x_1},\bar{x_2}))$ und enthält alle Tangenten an die Funktion $f(x_1,x_2)$ im Punkt $\bar{P}$.
\end{bemerkung}

\begin{satz}
Es seien $\mathbb{D} \subset \mathbb{R}^2, f:\mathbb{D}->\mathbb{R}$ eine partiell differenzierbare Funktion und $(x_{01},x_{02})\in\mathbb{D}$. Dann lautet die Gleichung der Tangentialebene für den Punkt $(x_{01},x_{02})$:

\[x_3=f(x_{01},x_{02})+\frac{\partial f}{\partial x_1}(x_{01},x_{02})(x_1-x_{01})+\frac{\partial f}{\partial x_2}(x_{01},x_{02})(x_2-x_{02})\]

\[=f(x_{01},x_{02})+\begin{pmatrix} \frac{\partial f}{\partial x_1} (x_{01},x_{02}) & \frac{\partial f}{\partial x_2}(x_{01},x_{02}) \end{pmatrix} \cdot \begin{pmatrix} (x_1-x_{01})\\ (x_2-x_{02}) \end{pmatrix}\]
\end{satz}

\begin{beispiel}
	Gegeben sei $f:\mathbb{R}^2->\mathbb{R}$ durch $f(x_1,x_2)=\sin(x_1\cdot x_2^2)$

	$\frac{\partial f}{\partial x_1}(x_1,x_2)=\cos(x_1\cdot x_2^2)\cdot x_2^2$
	
	$\frac{\partial f}{\partial x_2}(x_1,x_2)=\cos(x_1\cdot x_2^2)\cdot 2x_2 \cdot x_1$
	
	$\to$ Tangentialebene: $x_3 = f(x_{01},x_{02}) + \cos(x_1\cdot x_2^2)\cdot x_2^2 \cdot (x_1 - x_{01}) + \cos(x_1\cdot x_2^2)\cdot 2x_2 \cdot x_1 \cdot (x_2 - x_{02})$
	
	$(x_{01},x_{02}) = (0,0)$
	
	$x_3 = 0 + \begin{pmatrix}0 & 0\end{pmatrix} \begin{pmatrix}x_1 - 0 \\ x_2 - 0\end{pmatrix} = 0$
	
\end{beispiel}

\begin{beispiel}
	
	$(x_{01},x_{02}) = (\sqrt[3]{\pi},\sqrt[3]{\pi})$
	
	$f(\sqrt[3]{\pi},\sqrt[3]{\pi})=\sin(\sqrt[3]{\pi}\cdot \sqrt[3]{\pi}^2) = 0$
	
	$\frac{\partial f}{\partial x_1}(\sqrt[3]{\pi},\sqrt[3]{\pi})=\cos(\sqrt[3]{\pi}\cdot \sqrt[3]{\pi}^2)\cdot \sqrt[3]{\pi}^2 = -\pi^{\frac{2}{3}}$
	
	$\frac{\partial f}{\partial x_2}(\sqrt[3]{\pi},\sqrt[3]{\pi})=\cos(\sqrt[3]{\pi}\cdot \sqrt[3]{\pi}^2)\cdot 2\sqrt[3]{\pi} \cdot \sqrt[3]{\pi} = -2\pi^{\frac{2}{3}}$
	
	$x_3 = 0 + \begin{pmatrix}-\pi^{\frac{2}{3}} & -2\pi^{\frac{2}{3}}\end{pmatrix} \begin{pmatrix}x_1 - \sqrt[3]{\pi} \\ x_2 - \sqrt[3]{\pi}\end{pmatrix} = -\pi^{\frac{2}{3}} \cdot (x_1 - \sqrt[3]{\pi}) + -2\pi^{\frac{2}{3}} \cdot (x_2 - \sqrt[3]{\pi})$
	
	$x_3 = -\pi^{\frac{2}{3}}x_1 - 2\pi^{\frac{2}{3}}x_2 + 3\pi$
\end{beispiel}

\begin{beispiel}
	Gegeben sei $f:\mathbb{R}^2\to\mathbb{R}$ durch $f(x_1,x_2)=\left| x_1 \right| + x_2$
	$\frac{\partial f}{\partial x_2}(x_1,x_2)=1$
	$\left| x_1 \right|$ ist nur für $x_1 \neq 0$ differenzierbar, d.h. f ist nicht auf dem gesamten Definitionsbereich partiell differenzierbar.
\end{beispiel}

\begin{defi}[Gradient]
	Es sei $\mathbb{D} \subset \mathbb{R}^n$ und $f:\mathbb{D} \to \mathbb{R}$ partiell differenzierbar. Dann heißt der Vektor $\mathrm{grad} f(x) = \begin{pmatrix}
	\frac{\partial f}{\partial x_1}(x) \\
	\frac{\partial f}{\partial x_2}(x) \\
	\vdots \\
	\frac{\partial f}{\partial x_n}(x) \\
	\end{pmatrix}$ der Gradient von $f$ im Punkt $x\in\mathbb{D}=(x_1,\dots,x_n)$.
\end{defi}

\begin{bemerkung}
	Anstelle von $\mathrm{grad} f(x)$ wird auch häufig $\nabla f(x)$ geschrieben.
\end{bemerkung}

\begin{beispiel}
	Berechnen Sie den Gradienten für:
	\begin{enumerate}
		\item $f(x_1,x_2) = x_1^2 + x_2^2$
		\item $g(x_1,x_2,x_3) = 2 \cdot \sin(x_1 x_2) + x_1 x_2 x_3$
	\end{enumerate}

	zu 1. $\nabla f(x_1,x_2) = \begin{pmatrix}2x_1\\2x_2\end{pmatrix}$
	
	zu 2. $\nabla g(x_1,x_2,x_3) = \begin{pmatrix}2x_2 \cos(x_1 x_2)+x_2 x_3 \\ 2x_1 \cos(x_1 x_2)+x_1 x_3 \\ x_1 x_2\end{pmatrix}$
\end{beispiel}

\begin{beispiel}
	Partielle Differenzierbarkeit impliziert nicht Stetigkeit.
	
	Betrachtet wird die Funktion $f:\mathbb{R}^2\to\mathbb{R}$ mit
	
	\[f(x_1,x_2) = \begin{cases}
	\frac{x_1\cdot x_2}{x_1^2 + x_2^2} & (x_1,x_2) \ne (0,0) \\
	0 & (x_1,x_2) = (0,0)
	\end{cases}\]
	
	Im Punkt $(0,0)$ existieren die partiellen Ableitungen:
	
	\[\frac{\partial f}{\partial x_1}(0,0) = \lim_{\Delta x\to0}{\frac{f(0+\Delta x,0)-f(0,0)}{\Delta x}} = 0\]
	
	\[\frac{\partial f}{\partial x_2}(0,0) = \lim_{\Delta x\to0}{\frac{f(0,0+\Delta x)-f(0,0)}{\Delta x}} = 0\]
	
	Aber $f$ ist in $(0,0)$ nicht stetig:
	
	Es gilt $f(x_1,0) = 0; x_1 \in \mathbb{R}$, $f(0,x_2) = 0; x_2 \in \mathbb{R}$
	
	Für $x := x_1 = x_2$:
	
	$f(x,x)=\begin{cases}\frac{x^2}{2x^2}=\frac{1}{2} & (x,x) \ne (0,0) \\ 0 & (x,x) = (0,0)\end{cases}$
\end{beispiel}

\begin{defi}[Stetigkeit]
	Die Funktion $y=f(x_1,\dots,x_n)$, $(x_1,\dots,x_n)\in\mathbb{D}$ ist an der Stelle $\bar{P}=(\bar{x_1},\dots,\bar{x_n}) \in \mathbb{D}$ stetig, wenn für den Funktionsgrenzwert
	
	\[\lim_{P\to\bar{P}}{f(x_1,\dots,x_n)} = f(\bar{x_1},\dots,\bar{x_n})\] gilt.
\end{defi}

Aufgaben:

\begin{enumerate}
	\item Berechnen Sie alle ersten und zweiten partiellen Ableitungen der Funktion $f(x_1,x_2)=e^{-2.5x_1^2-(x_2-1)^2}, (x_1,x_2) \in \mathbb{R}$
	\item Berechnen die den Gradienten der Funktion $f(x_1,x_2,x_3)=\frac{(x_1-1)\cdot\ln(x_1+1)}{x_2^2+x_3^2+1}$ an der Stelle $(0,0,0)$
	\item Berechnen Sie die Tangentialebene an die Funktion  $f(x_1,x_2)=e^{-2.5x_1^2-(x_2-1)^2}, (x_1,x_2) \in \mathbb{R}$ im Punkt $(0,\frac{3}{2},e^{-\frac{1}{4}})$
\end{enumerate}

\begin{defi}[k-mal partiell differenzierbar]
	Es sei $D\subset\mathbb{R}^n,f:D\to\mathbb{R}$ eine partiell differenzierbare Funktion und $x_0 \in D$. Die Funktion f heißt zweimal \underline{partiell differenzierbar} in $x_0$, wenn alle partiellen Ableitungen $\frac{\partial f}{\partial x_i}$ in $x_0$ wieder partiell differenzierbar sind.

	Man schreibt $\frac{\partial^2 f}{\partial x_jx_i}(x_0)=\frac{\partial}{\partial x_j}\left(\frac{\partial f}{\partial x_i}\right)(x_0)=f_{x_jx_i}(x_0)$

	Dieser Ausdruck heißt dann \underline{zweite partielle Ableitung} von f.

	Allgemein heißt f \underline{k-mal partiell differenzierbar}, wennn alle (k-1)ten partiellen Ableitungen von f wieder partiell differenzierbar sind. Man schreibt:

	\[\frac{\partial f}{\partial x_{ik} \partial x_{i(k-1)} \dots \partial x_{i1}}(x_0) = \frac{\partial}{\partial x_{ik}}\left(\frac{\partial^{k-1} f}{\partial x_{i(k-1)} \dots \partial x_{i1}}\right)(x_0)=f_{x_{ik} \dots x_{i1}}\]
\end{defi}

\begin{aufg}
	Gegeben sei $f:\mathbb{R}^2\to\mathbb{R}$ durch $f(x_1,x_2)=x_1^2 \cdot x_2 - x_1 \cdot x_2^2$

	Berechnen Sie alle ersten und zweiten Ableitungen von f und $\frac{\partial^3 f}{\partial x_1 \partial x_1 \partial x_2}(x_1,x_2)$.

	\[\frac{\partial f}{\partial x_1}(x_1,x_2)=2x_1 \cdot x_2 - x_2^2 \]
	\[\frac{\partial f}{\partial x_2}(x_1,x_2)=x_1^2 - x_1 \cdot 2x_2 \]

	\[\frac{\partial^2 f}{\partial x_1^2}(x_1,x_2)=2x_2 ; \frac{\partial^2 f}{\partial x_2^2}(x_1,x_2)=-2x_1 \]
	\[\frac{\partial^2 f}{\partial x_1x_2}(x_1,x_2)=2x_1-2x_2 ; \frac{\partial^2 f}{\partial x_2x_1}(x_1,x_2)=2x_1-2x_2 \]
	\[\frac{\partial^3 f}{\partial x_1 \partial x_1 \partial x_2}(x_1,x_2)=+2\]
\end{aufg}

\begin{defi}[stetig partiell differenzierbar]
	Es sei $D\subset\mathbb{R}^n,f:D\to\mathbb{R}$. f heißt k-mal \underline{stetig partiell differenzierbar}, falls f k-mal partiell differenzierbar ist und alle partiellen Ableitungen der Ordnung k stetig sind.
\end{defi}

\begin{satz}[Satz von Schwarz]
	Es sei $D\subset\mathbb{R}^n,f:D\to\mathbb{R}$ zweimal stetig partiell differenzierbar. Dann ist $\frac{\partial^2 f}{\partial x_i \partial x_j}=\frac{\partial^2 f}{\partial x_j \partial x_i}$ für alle $i,j\in\left\{1,\dots,n\right\}$. Die Reihenfolge der Ableitungsvariablen spielt also keine Rolle,
\end{satz}

\begin{bemerkung}
	Es sei $f:D\to\mathbb{R}$ k-mal stetig partiell differenzierbar. Dann spielt die Reihenfolge der Ableitungsvariablen bei der k-ten partiellen Ableitung keine Rolle.
\end{bemerkung}

\begin{aufg}
	Berechnen Sie alle ersten und zweiten partiellen Ableitungen der Funktion $f(x_1,x_2,x_3)=x_3 \cdot \sin(x_1^2+x_2)+e^{2x_3}$ $(x_1,x_2,x_3)\in\mathbb{R}^3$

	\(\frac{\partial f}{\partial x_1}(x_1,x_2,x_3) = x_3 \cdot 2x_1 \cdot \cos(x_1^2+x_2)\)

	\(\frac{\partial f}{\partial x_2}(x_1,x_2,x_3) = x_3 \cdot \cos(x_1^2+x_2)\)

	\(\frac{\partial f}{\partial x_3}(x_1,x_2,x_3) = \sin(x_1^2+x_2) + 2e^{2x_3}\)

	\(\frac{\partial^2 f}{\partial x_1^2}(x_1,x_2,x_3) = 2x_3 \left(\cos(x_1^2+x_2) - 2x_1^2\sin(x_1^2+x_2)\right)\)

	\(\frac{\partial^2 f}{\partial x_1 \partial x_2}(x_1,x_2,x_3) = -2x_1x_3 \sin(x_1^2+x_2)\)

	\(\frac{\partial^2 f}{\partial x_2^2}(x_1,x_2,x_3) = -x_3\sin(x_1^2+x_2)\)

	\(\frac{\partial^2 f}{\partial x_1 \partial x_3}(x_1,x_2,x_3) = 2x_1 \cos(x_1^2+x_2)\)

	\(\frac{\partial^2 f}{\partial x_2 \partial x_3}(x_1,x_2,x_3) = \cos(x_1^2+x_2)\)

	\(\frac{\partial^2 f}{\partial x_3^2}(x_1,x_2,x_3) = 4e^{2x_3}\)
\end{aufg}

\section{Totale Differenzierbarkeit}

\begin{defi}[Betrag eines Vektors]
	Der Betrag eines Vektors $x=(x_1,\dots,x_n)\in\mathbb{R}^n$ ist definiert als:
	\[\vert x \vert = \sqrt{x_1^2+\cdots+x_n^2}\]
\end{defi}

\begin{defi}[Vektorfunktion]
	Eine eindeutige Abbildung $f:\mathbb{D}\to W,\mathbb{D}\subset\mathbb{R}^n,W\subset\mathbb{R}^m,m>1$ mit mehrdimensionalem Wertebereich heißt \underline{Vektorfunktion}.
\end{defi}

\begin{beispiel}
	Es sei $\mathbb{D}\subset\mathbb{R}^3\to\mathbb{R}^2$ gegeben durch
	\[f(x_1,x_2,x_3)=\begin{pmatrix}
	x_1+x_3 \\ x_2+x_3
	\end{pmatrix}\]
	f hat 2 Ergebniskomponenten:
	\[f_1(x_1,x_2,x_3)=x_1+x_2,f_2(x_1,x_2,x_3)=x_2+x_3\]
\end{beispiel}

\begin{defi}[total differenzierbar]
	Es sei $f:\mathbb{D}\to \mathbb{R}^m,\mathbb{D}\subset\mathbb{R}^n$ eine Abbildung und $x_0\in\mathbb{D}$. Die Funktion f heißt total differenzierbar in $x_0$, falls es eine Matrix $A\in\mathbb{R}^{m\times n}$ und eine Restfunktion $R:\mathbb{D}\to\mathbb{R}^m$ gibt, für die gilt: $f(x)=f(x_0)+A(x-x_0)+\vert x-x_0 \vert\cdot R(x)$ und \[\lim_{x\to x_0} R(x)=0\]
\end{defi}

\begin{satz}[Jacobi-Matrix]
	Es sei $f:\mathbb{D}\to \mathbb{R}^m,\mathbb{D}\subset\mathbb{R}^n$ eine Abbildung und $x_0\in\mathbb{D}$. Weiterhin sei f in $x_0$ total differenzierbar mit der Matrix
	\[A=(a_{ij});i=1,\dots,m;j=1,\dots,n \in\mathbb{R}^{m\times n}\]
	Dann ist f in $x_0$ stetig und alle Komponentenfunktionen $f_1,\dots,f_m:\mathbb{R}^n\to\mathbb{R}$ sind in $x_0$ partiell differenzierbar, wobei gilt: $a_{ij}=\frac{\partial f_i}{\partial x_j}(x_0)$.
	
	\[A=\begin{pmatrix}
	\frac{\partial f_1}{\partial x_1}(x_0) & \frac{\partial f_1}{\partial x_2}(x_0) & \cdots & \frac{\partial f_1}{\partial x_n}(x_0) \\
	\frac{\partial f_2}{\partial x_1}(x_0) & \frac{\partial f_2}{\partial x_2}(x_0) & \cdots & \frac{\partial f_2}{\partial x_n}(x_0) \\
	\vdots & \vdots & \ddots & \vdots \\
	\frac{\partial f_m}{\partial x_1}(x_0) & \frac{\partial f_m}{\partial x_2}(x_0) & \cdots & \frac{\partial f_m}{\partial x_n}(x_0)
	\end{pmatrix}\]
	
	Die Matrix heißt Funktionalmatrix oder auch Jacobi-Matrix von f und wird mit $Df(x_0)$ oder $J_f(x_0)$ bezeichnet.
\end{satz}

\begin{beispiel}
	Gegeben sei die Funktion $g:\mathbb{R}\to\mathbb{R}^2$ durch \[g(x)=\begin{pmatrix}2x^3 \\ \ln(x^2+1)
	\end{pmatrix} = \begin{pmatrix}
	g_1(x) \\ g_2(x)
	\end{pmatrix}\]
	Berechnen Sie die Jacobi-Matrix $D_g(x)$:
	\[D_g(x)=\begin{pmatrix}6x^2 \\ \frac{1}{x^2+1}\cdot2x\end{pmatrix}\]
\end{beispiel}

\begin{aufg}
	\[f(x)=\begin{pmatrix}
	x_1 + x_2 \\
	x_1^2\cdot\sin x_2 \\
	e^{x_1x_2}
	\end{pmatrix}\]
	\[D_f(x_1,x_2)=?\]
\end{aufg}

\begin{bemerkung}
	Nach Definition 1.2.4 kann eine Funktion f(x), die total differenzierbar ist, in der Nähe von $x_0$ durch \[f(x_0)+D_f(x_0)(x-x_0)\] angenähert werden.
	
	Da die Annäherungsfunktion linear ist, spricht man auch von Linearisierung.
	
	Hierfür muss aber klar sein, dass f auch wirklich total differenzierbar ist.
\end{bemerkung}

\begin{satz}
	Es sei $f:\mathbb{D}\to \mathbb{R}^m,\mathbb{D}\subset\mathbb{R}^n$ eine Abbildung, deren Komponentenfunktionen $f_1,\dots,f_m$ alle stetig partiell differenzierbar sind. Dann ist f total differenzierbar.
\end{satz}

\begin{beispiel}
	Gegeben sei $f:\mathbb{R}^3\to\mathbb{R}$ durch
	\[f(x_1,x_2,x_3)=x_1^2x_2+x_1x_2\sin(x_3)\]
	$Df(x_1,x_2,x_3)=\begin{pmatrix}
	2x_1x_2+x_2\sin(x_3) & x_1^2+x_1\sin(x_3) & x_1x_2\cos(x_3)
	\end{pmatrix}$
	
	f soll in der Nähe des Punktes $x_0=(x_{01},x_{02},x_{03})=(1,1,\frac{\pi}{2})$ angenähert werden. Zur Berechnung der Näherung wird benötigt: $f(x_0)+Df(x_0)(x-x_0)$
	
	$f(x_0)=1+1\cdot1=2$
	
	$Df(x_0)=\begin{pmatrix}3 & 2 & 0\end{pmatrix}$
	
	$2+\begin{pmatrix}3 & 2 & 0\end{pmatrix}\cdot\begin{pmatrix}
	x_1 - 1 \\
	x_2 - 1 \\
	x_3 - \frac{\pi}{2}
	\end{pmatrix}$
	
	$f(x_1,x_2,x_3)=3x_1+2x_2-3$
\end{beispiel}

\begin{aufg}
Gegeben sei die Funktion $f:\mathbb{R}^2\to\mathbb{R}^3$ durch $f(x_1,x_2)=\begin{pmatrix}x_1+x_2 & x_1^2\sin(x_2) & e^{x_1x_2}\end{pmatrix}^T$ (siehe Aufgabe 1.2.7)

	Berechnen Sie die Näherung der Funktion in der Nähe von $x_0=(1,0)$
\end{aufg}

\begin{aufg}
	Berechnen Sie die Tangentialebenen an die Funktion $f(x_1,x_2)=e^{-2.5x_1^2-(x_3-1)^2}$ $(x_1,x_2)\in\mathbb{R}^2$ im Punkt $(x_{01},x_{02},f(x_{01},x_{02}))=(0,1,1)$.

	Verwenden Sie dazu zunächst die Formel für die Tangentialebenen und dann die Formel für die totale Differenzierbarkeit.
\end{aufg}

\begin{defi}[totale Differenzierbarkeit]
Sei $f:\mathbb{R}^2\to\mathbb{R}$ eine total differenzierbare Funktion. Dann kann f in der Nähe eines Punktes $(x_{01},x_{02})\in\mathbb{R}^2$ durch $f(x_{01},x_{02})+Df(x_{01},x_{02})\begin{pmatrix}x_1-x_{01} \\ x_2-x_{02}\end{pmatrix}$ angenähert werden.

	Es gilt also $f(x_1,x_2)=f(x_{01},x_{02})+\begin{pmatrix}
		\frac{\partial f}{\partial x_1}(x_{01},x_{02}) & \frac{\partial f}{\partial x_2}(x_{01},x_{02})\end{pmatrix}\begin{pmatrix}
			x_1-x_{01} \\ x_2-x_{02}
		\end{pmatrix}$

\[f(x_1,x_2)-f(x_{01},x_{02})\approx\frac{\partial f}{\partial x_1}(x_{01},x_{02})\cdot\underbrace{(x_1-x_{01})}_{=:\Delta x_1} + \frac{\partial f}{\partial x_2}(x_{01},x_{02})\cdot\underbrace{(x_2-x_{02})}_{=:\Delta x_2}\]

also gilt: $\Delta f\approx \frac{\partial f}{\partial x_1}\cdot\Delta x_1 + \frac{\partial f}{\partial x_2}\cdot\Delta x_2$

Für \enquote{beliebig kleines} $\Delta x_1$ und $\Delta x_2$ schreiben wir \enquote{d} statt \enquote{$\Delta$} und \enquote{=} statt \enquote{$\approx$}.

\[\boxed{df = \frac{\partial f}{\partial x_1}(x_{01},x_{02})\cdot dx_1 + \frac{\partial f}{\partial x_2}(x_{01},x_{02})\cdot dx_2} \]
Diesen Ausdruck bezeichnet man als \underline{totales Differenzial} der Funktion f.

Für eine Funktion $f:\mathbb{R}^n\to\mathbb{R}$ gilt:
\[\boxed{df = \frac{\partial f}{\partial x_1}\cdot dx_1 + \frac{\partial f}{\partial x_2}\cdot dx_2 + \cdots + \frac{\partial f}{\partial x_2}\cdot dx_n}\]
\end{defi}

\begin{bemerkung}
	Mit Hilfe des totalen Differentials kann der Einfluss der Änderung der Inputgrößen auf den Funktionswert abgeschätzt werden.
\end{bemerkung}

\begin{beispiel}
	Gegeben sei $f:\mathbb{R}^2\to\mathbb{R}$ durch $f(x_1,x_2)=x_1^2+\sin(x_1\cdot x_2)$. Wir betrachten f in der Nähe des Punktes $f(x_{01},x_{02})=(\sqrt{\pi},\sqrt{\pi})$

	Was passiert wenn wir leicht von dem Wert abweichen?

	\[\frac{\partial f}{\partial x_1}=2x_1+\cos(x_1\cdot x_2)x_2\]
	\[\frac{\partial f}{\partial x_2}=\cos(x_1 \cdot x_2)x_1\]

	\(df=2\sqrt{\pi}+\cos(\pi)\cdot\sqrt{\pi}\cdot(x_1-\sqrt{\pi})+\cos(\pi)\cdot\sqrt{\pi}\cdot(x_2-\sqrt{\pi}) \)
	\(df=\sqrt{\pi}\cdot dx_1+(-\sqrt{\pi})\cdot dx_2\)

	Veränderung der Inputgrößen z.B. $dx_1=0.1$ $dx_2=-0.1$

	$df=0.2\cdot\sqrt{\pi}$
\end{beispiel}

\begin{bemerkung}
	Durch Einsetzen der Maximalen absoluten Fehler und Bilden der Beträge ergibt sich der (lineare) maximale absolute Fehler:

	\[\left|\Delta f\right|\approx \left|\frac{\partial f}{\partial x_1}(x_1,\dots,x_n)\right|\cdot\left|\Delta x_1\right|+\dots+\left|\frac{\partial f}{\partial x_n}(x_1,\dots,x_n)\right|\cdot\left|\Delta x_n\right|\]
\end{bemerkung}

\begin{aufg}
	Das Volumen V eines geraden Kreiskegels wird berechnet durch: $V=\frac{\pi}{3}\cdot r^2 \cdot \sqrt{k^2-r^2}$.

	r\dots Radius r=1m und absoluter Fehler von 0.01m

	k\dots Mantellinie k=1.5m und der absolute Fehler: 0.005m

	Berechnen Sie den (linearen) maximalen absoluten Fehler.

	Papula zu Abschnitt 2 S. 332: 12, 13, 15
\end{aufg}

\section{Extremwerte}

\begin{defi}[Lokales Extremum]
	Es seien $U\subset\mathbb{R}^n$ eine offene Menge und $f:U\to\mathbb{R}$ eine Abbildung.
	\begin{enumerate}[label=(\roman*)]
		\item Ein Punkt $x_0\in U$ heißt \underline{lokales Maximum} von $f$, falls $f$ in der Nähe von $x_0$ nicht größer wird als bei $x_0$, das heißt:
		$f(x)\le f(x_0)$ für alle $x$ in der Nähe von $x_0$.
		\item Ein Punkt $x_0\in U$ heißt \underline{lokales Minimum} von $f$, falls $f$ in der Nähe von $x_0$ nicht kleiner wird als bei $x_0$, das heißt:
		$f(x)\ge f(x_0) $ für alle $x$ in der Nähe von $x_0$.
	\end{enumerate}
	Ein \underline{lokales Extremum} ist ein lokales Minimum oder ein lokales Maximum.
\end{defi}

\begin{bemerkung}
	Wie im eindimensionalen Fall liefert die Differentialrechnung nur Informationen über lokale und nicht über globale Extrema.
\end{bemerkung}

\begin{bemerkung}
	Es sei $f:(a,b)\to\mathbb{R}$ differenzierbar in $x_0\in (a,b)$ mit einem lokalen Extremum in $x_0$. Dann ist $f'(x_0)=0$.
\end{bemerkung}

\begin{satz}
	Es sei $U \subset \mathbb{R}^n$ eine offene Menge und $f:U\to\mathbb{R}$ differenzierbar. Besitzt f in $x_0\in U$ ein lokales Extremum, so gilt:
	\[\nabla f(x_0)=0,\]
	d.h. $\frac{\partial f}{\partial x_1}(x_0)=\cdots=\frac{\partial f}{\partial x_n}(x_0)=0$
\end{satz}

\begin{beispiel}
	$f:\mathbb{R}^2\to\mathbb{R}$\hspace{1cm}$f(x_1,x_2)=1-x_1^2-x_2^2$
	
	\[\frac{\partial f}{\partial x_1}(x_1,x_2)=-2x_1\hspace{1cm}\frac{\partial f}{\partial x_2}(x_1,x_2)=-2x_2\]
	
	$\to x_1=x_2=0$
	
	D.h. falls es ein Extremum gibt, dann liegt es bei (0,0).
\end{beispiel}

\begin{aufg}
	\[f:\mathbb{R}^2\to\mathbb{R}\hspace{1cm}f(x_1,x_2)=\sin(x_1)\cdot\sin(x_2)\]

	\[\nabla f(x_1,x_2)=\begin{pmatrix}
		\cos(x_1)\cdot\sin(x_2) \\
		\sin(x_1)\cdot\cos(x_2)
	\end{pmatrix} =: \begin{pmatrix}0 \\ 0\end{pmatrix}\]

		$\cos(x_1)\cdot\sin(x_2)=0$

		$(x_1=\pi)$ od. $\sin(x_2)=0$
\end{aufg}

\begin{bemerkung}
	Es sei $f:(a,b)\to\mathbb{R}$ zweimal differenzierbar und $f'(x_0)=0$.
	\begin{enumerate}[label=(\roman*)]
		\item Ist $f''(x_0)>0$, so hat f in $x_0$ ein lokales Minimum.
		\item Ist $f''(x_0)<0$, so hat f in $x_0$ ein lokales Maximum.
	\end{enumerate}
\end{bemerkung}

\begin{defi}
	Eine Matrix $A=(a_{ij}), i,j=1,\dots,n \in \mathbb{R}^{n\times n}$ heißt \underline{positiv definit}, falls gilt:
	$a_{11}>0,\det\begin{pmatrix}
	a_{11} & a_{12} \\ a_{21} & a_{22}
	\end{pmatrix} > 0, \dots, \det\begin{pmatrix}
	a_{11} & \cdots & a_{1n} \\
	a_{21} &  & a_{2n} \\
	\vdots & & \vdots \\
	a_{n1} & \cdots & a_{nn}
	\end{pmatrix} > 0$, also $\det\begin{pmatrix}
	a_{11} & \cdots & a_{1k} \\
	a_{21} &  & a_{2k} \\
	\vdots & & \vdots \\
	a_{k1} & \cdots & a_{kk}
	\end{pmatrix} > 0$ für alle $k=1,\dots,n$.
	
	A heißt \underline{negativ definit}, falls -A positiv definit ist.
\end{defi}

\begin{beispiel}
	Es sei $A=\begin{pmatrix}
	1 & 1 \\ 1 & 4
	\end{pmatrix} \in \mathbb{R}^{2\times 2}$
	
	$\underline{k=1}:1>0$
	
	$\underline{k=2}: \det\begin{pmatrix}
	1 & 1 \\ 1 & 4
	\end{pmatrix} = 1\cdot4-1\cdot1=3>0$
	
	$\to$ A ist positiv definit.
\end{beispiel}

\begin{aufg}
	$B=\begin{pmatrix}
	-6 & 2 \\ 2 & -1
	\end{pmatrix} \in \mathbb{R}^{2\times 2}$
	
	$-6<0 \to$ B nicht positiv definit
	
	$-B=\begin{pmatrix}
	6 & -2 \\ -2 & 1
	\end{pmatrix}$
	
	$\underline{k=1}:6>0$
	
	$\underline{k=2}: \det\begin{pmatrix}
	6 & -2 \\ -2 & 1
	\end{pmatrix} = 6\cdot1 - (-2)\cdot(-2)=2>0$
	
	$\to$ B ist negativ definit.
\end{aufg}

\begin{aufg}
	Prüfen Sie $C=\begin{pmatrix}
	1 & 1 & 0 \\ 1 & 3 & 1 \\ 0 & 1 & 2
	\end{pmatrix} \in \mathbb{R}^{3\times3}$ auf positive Definitheit.
\end{aufg}

\begin{bemerkung}
	$D=\begin{pmatrix}
	0 & 1 \\ 1 & 4
	\end{pmatrix} \in \mathbb{D}^{2\times2}$ ist weder positiv noch negativ definit: $d_{11}\not>0\to$ nicht positiv definit
	
	$-d_{11}\not>0\to$ nicht negativ definit
\end{bemerkung}

\begin{defi}
	Es seien $U\subset\mathbb{R}^n$ eine offene Menge, $f:U\to\mathbb{R}$ eine zweimal stetig partiell differenzierbare Funktion und $x_0\in U$. Unter der \underline{Hesse-Matrix} von f in $x_0$ versteht man die Matrix: \[H_f(x_0)=\begin{pmatrix}
	\frac{\partial^2 f}{\partial x_1^2}(x_0) & \cdots & \frac{\partial^2 f}{\partial x_1 \partial x_n}(x_0) \\
	\frac{\partial^2 f}{\partial x_2 \partial x_1}(x_0) & \cdots & \frac{\partial^2 f}{\partial x_2 \partial x_n}(x_0) \\
	\vdots & \ddots & \vdots \\
	\frac{\partial^2 f}{\partial x_n \partial x_1}(x_0) & \cdots & \frac{\partial^2 f}{\partial x_n^2}(x_0) \\
	\end{pmatrix}\]
\end{defi}

\begin{beispiel}
	Gegeben sei $f:\mathbb{R}^2\to\mathbb{R}$ durch $f(x_1,x_2)=x_1^2+x_2^2$. Gesucht ist die Hesse-Matrix.
	
	\[H_f(x_1,x_2)=\begin{pmatrix}
	\frac{\partial^2 f}{\partial x_1^2}(x_1,x_2)  & \frac{\partial^2 f}{\partial x_1 \partial x_2}(x_1,x_2) \\
	\frac{\partial^2 f}{\partial x_2 \partial x_1}(x_1,x_2) & \frac{\partial^2 f}{\partial x_2^2}(x_1,x_2)
	\end{pmatrix}=\begin{pmatrix}
	2 & 0 \\ 0 & 2
	\end{pmatrix}\]
\end{beispiel}

\begin{aufg}
	Gegeben sei $f:\mathbb{R}^3\to\mathbb{R}$ durch
	\[f(x_1,x_2,x_3)=x_1^3 \cdot x_2^3 \cdot \sin(x_3)\]
	Berechnen Sie die Hesse-Matrix $H_f(0,0,0)$ \& $H_f(1,1,0)$.
\end{aufg}

\begin{satz}
	Es sei $U\subset\mathbb{R}^n$ eine offene Menge, $f:U\to\mathbb{R}$ zweimal stetig differenzierbar und $x_0\in U$ ein Punkt mit $\nabla f(x_0)=0$.
	\begin{enumerate}[label=(\roman*)]
		\item Ist $H_f(x_0)$ positiv definit, so hat f in $x_0$ ein lokales Minimum.
		\item Ist $H_f(x_0)$ negativ definit, so hat f in $x_0$ ein lokales Maximum.
		\item Ist $\underline{U\subset\mathbb{R}^2}$ und gilt $\det H_f(x_0) < 0$, so liegt kein Extremwert vor.
	\end{enumerate}
\end{satz}

\begin{beispiel}
	Gegeben sei $f:\mathbb{R}^2\to\mathbb{R}$ durch $f(x_1,x_2)=1+x_1^2+x_2^2$
	
	$\nabla f(x_1,x_2)=\begin{pmatrix}
	2x_1 \\ 2x_2
	\end{pmatrix} := 0 \to x_1=x_2=0$
	
	$\to (0,0)$ könnte ein Extremum sein
	
	Überprüfen durch Hesse-Matrix
	
	$H_f(x_1,x_2)=\begin{pmatrix}
	2 & 0 \\ 0 & 2
	\end{pmatrix}$
	
	$2>0, \det\begin{pmatrix}
	2 & 0 \\ 0 & 2
	\end{pmatrix}=4>0$
	
	$\to H_f(0,0)$ positiv definit $\to$ in $(0,0)$ liegt ein lokales Minimum vor.
\end{beispiel}

\begin{aufg}
	Gegeben sei $f:\mathbb{R}^2\to\mathbb{R}$ durch $f(x_1,x_2)=\sin(x_1)\cdot\sin(x_2)$
	
	Untersuchen Sie die Funktion auf lokale Extrema.
	
	Kandidaten: $(0,0),(\frac{\pi}{2},\frac{\pi}{2})$
\end{aufg}

\begin{aufg}
	Gegeben sei die Funktion $f:\mathbb{R}^2\to\mathbb{R}$ durch $f(x_1,x_2)=\cos(x_1)+\cos(x_2)$
	
	Gradient: $\nabla f(x_1,x_2)=\begin{pmatrix}
	-\sin x_1 \\ -\sin x_2
	\end{pmatrix} := \begin{pmatrix}
	0 \\ 0
	\end{pmatrix} \to -\sin x_1 = -\sin x_2 = 0$
	
	$(k_1\pi,k_2\pi), k_1,k_2\in\mathbb{Z}$
	
	Hesse-Matrix
	
	$H_f(x_1,x_2)=\begin{pmatrix}
	-\cos x_1 & 0 \\ 0 & -\cos x_2
	\end{pmatrix}$
	
	$H_f(k_1\pi,k_2\pi)=\begin{pmatrix}
	-(-1)^{k_1} & 0 \\ 0 & -(-1)^{k_2} 
	\end{pmatrix}$
	
	\vspace{1cm}
	\def\arraystretch{1.25}	
	\begin{tabular}{|c|c|c|} \hline
	& $k_1$ gerade & $k_1$ ungerade \\ \hline
	$k_2$ gerade & lokales Maximum & kein Extremwert \\ \hline
	$k_2$ ungerade & kein Extremwert & lokales Minimum \\ \hline
	\end{tabular}
\end{aufg}

\begin{beispiel}[Nebenbedingungen]
	Gegeben sei 12m langer Draht, aus dem die Kanten eines Quaders von möglichst großem Volumen hergestellt werden sollen. Gesucht sind die Kantenlängen $x_1,x_2,x_3$ des optimalen Quaders.
	
	$4x_1+4x_2+4x_3=4(x_1+x_2+x_3)=12$
	
	$x_1+x_2+x_3=3$
	
	$V=x_1\cdot x_2 \cdot x_3$
	
	$x_1,x_2,x_3 > 0$
	
	$V=x_1 x_2 (3-x_1-x_2)$
	
	$V=3x_1x_2-x_1^2x_2-x_1x_2^2$
	
	$\mathbb{D}={(x_1,x_2)\in\mathbb{R}^2:x_1>0,x_2>0,x_1+x_2<3}$, $\mathbb{D}$ ist eine offene Menge
	
	$\nabla V(x_1,x_2)=\begin{pmatrix}
	3x_2-2x_1x_2-x_2^2 \\
	3x_1-x_1^2-2x_1x_2
	\end{pmatrix} := \begin{pmatrix}
	0 \\ 0
	\end{pmatrix}$
	
	$\to x_1=1, x_2=1, x_3=1$
	
	Berechnen Sie die Hesse-Matrix

	Beispiel ***:
	
	\[f:\mathbb{R}^2\to\mathbb{R} f(x,y)=\begin{cases}
		x \cdot y\cdot\frac{x^2-y^2}{x^2+y^2} & (x,y)\ne(0,0) \\
		0 & (x,y)=(0,0)
	\end{cases}\]

	(a) Stetigkeit in (0,0)

	\[\bar{P} \lim_{P\to\bar{P}}f(x_1,\dots,x_n)=f(\bar{x_1},\dots,\bar{x_n})\]

	Es gilt: $\lim_{(x,y)\to(0,0)} f(x,y)=\lim_{(x,y)\to(0,0)} |x\cdot y\cdot\frac{x^2-y^2}{x^2+y^2}|=\lim_{(x,y)\to(0,0)} \underbrace{|x\cdot y|}_{\to 0}\cdot\underbrace{|\frac{x^2-y^2}{x^2+y^2}|}_{\leq 1}$
\end{beispiel}

\begin{aufg}
	Papula S.332 zu Abschnitt 2 $\to$ Aufg. 24
	
	\begin{enumerate}
		\item $f(x_1,x_2)=x_1^2(1-x_2)-x_2^3+12x_2+13$
		\item $f(x_1,x_2)=(x_1-1)^2(1-x_2)-x_2^3+12x_2+3$
		\item $f(x_1,x_2)=4(x_1^2-25)(x_2-2)+5x_2^2+12x_2$
	\end{enumerate}
\end{aufg}

\chapter{Grundlagen der Integralrechnung reeller Funktionenmit mehreren Variablen}

	\section{Zweidimensionale Integralrechnung}

	\begin{defi}[beschränkt]
		Eine Menge $U\subset\mathbb{R}^2$ heißt \underline{beschränkt}, wenn es ein Rechteck R gibt, sodass $U\subset R$ gilt.
	\end{defi}

	\begin{bemerkung}
		%\begin{tikzpicture}
		%	\begin{axis}
		%		\addplot3[] coordinates {
		%			(0,0,0) (0,0.5,0) (0,1,0) (0,1.5,0) (0,2,0)
		%			(1,0,0) (1,0.5,0) (1,1,0) (1,1.5,0) (1,2,0)
		%			(2,0,0) (2,0.5,0) (2,1,0) (2,1.5,0) (2,2,0)
		%		}
		%	\end{axis}
		%\end{tikzpicture}
	\end{bemerkung}
	
	\begin{bemerkung}
		Es sei $U\subset \mathbb{R}^2$ eine beschränkte Menge und $f:U\to \mathbb{R}$ eine stetige Funktion. Da ein Volumen betrachtet wird, werden die alten Näherungsrechtecke durch Näherungsquader ersetzt und deren Volumen zusammengezählt.

		U wird also in n kleine Teilbereiche $u_1,\dots,u_n$ zerlegt. Die Fläche dieser Teilbereiche wird mit $\delta u_1,\dots,\Delta u_n$ bezeichnet.

		Zur Berechnung des Rauminhalts des Quaders wird weiterhin die Quaderhöhe benötigt. Dazu wird ein Punkt $(x_i,y_i)\in U_i$ gewählt und sein Funktionswert $f(x_i,y_i)$ als Höhe des Quaders betrachtet. Das Teilvolumen beträgt dann $f(x_i,y_i)\cdot\Delta u_i$

		Falls $U_i$ ein Rechteck ist mit den Seiten $\Delta x_i$ und $\Delta y_i$, so ergibt sich das Teilvolumen
		
		\[\Delta V_i = f(x_i,y_i)\cdot \Delta U_i=f(x_i,y_i)\cdot \Delta x_i\cdot \Delta y_i\]

		Als Näherung für das Gesamtvolumen eribt sich also

		\[V_n(f)=\sum^{n}_{i=1} f(x_i,y_i)\cdot\Delta U_i\]

		Hier wurde der gesamte Definitionsbereich U in die n Teilbereiche $U_1,\dots,U_n$ zerlegt.

		Der genaue Wert für das Volumen kann berechnet werden, indem n gegen $\infty$ geht.

		Deshalb wird definiert:

		\[\int_{U} f(x,y)dU=\int_{U} f(x,y)dxdy\]

		\[= \lim_{n\to\infty} \sum^{n}_{i=1} f(x_i,y_i)\cdot\Delta U_i\]

		Um klarzustellen, dass es sich um ein zweidimensionales Integral handelt, werden oft die zwei Integralsymbole verwendet:

		\[\iint_{U} f(x,y)dU=\iint_{U} f(x,y)dxdy\]
	\end{bemerkung}

	\begin{defi}[konvex]
		Eine Menge $U\subset\mathbb{R}^2$ oder $U\subset\mathbb{R}^3$ heißt \underline{konvex}, falls für alle Punkte $x,y\in U$ auch die gesamte Verbindungsstrecke von x nach y in U liegt.
	\end{defi}

	\begin{bemerkung}
		% Skizze

		Es sei $U\subset\mathbb{R}^2$ eine beschränkte und konvexe Menge und $f:U\to\mathbb{R}$ eine stetige Funktion. Da U beschränkt ist, gibt es einen kleinsten vorkommenden x-Wert a und einen größten vorkommenden x-Wert b.

		Der Flächeninhalt der Schnittfläche des Körpers bei einem beliebigen x-Wert zwischen a und b wird mit $I(x)$ bezeichnet.

		Der Körper, dessen Volumen wir ausrechnen, setzt sich aus all diesen Schnittflächen zusammen. Für jedes $x\in [a,b]$ existiert eine Schnittfläche mit Flächeninhalt $I(x)$, das heißt durch Aufsummieren dieser unendlich vielen Flächeninhalte ergibt sich das Volumen des Körpers.

		\[\iint_U f(x,y)dxdy = \int^a_b I(x)dx\]

		Der Flächeninhalt von I(x) kann einfach mit einem eindimensionalen Integral berechnet werden.

		\[I(x) = \int^{y_o(x)}_{y_u(x)} f(x,y)dy \]

		Insgesamt ergibt sich also:

		\[\iint_U f(x,y)dxdy = \int^b_a \left(\int^{y_o(x)}_{y_u(x)} f(x,y)dy\right)dx\]
	\end{bemerkung}

	\begin{satz}
		Es sei $U\subset\mathbb{R}^2$ eine beschränkte und konvexe Menge und $f:U\to\mathbb{R}$ eine stetige Funktion. Weiterhin sei a der kleinste in U vorkommende x-Wert und b der größete in U vorkommende x-Wert. Für $x\in [a,b]$ bezeichnen wir den kleinsten y-Wert für den $(x,y) \in U$ gilt, als $y_u(x)$ und den größten y-Wert als $y_o(x)$.

		Dann ist $\iint_U f(x,y)dxdy = \int^b_a \left(\int^{y_o(x)}_{y_u(x)} f(x,y)dy\right)dx$
	\end{satz}

	\begin{beispiel}
		\begin{enumerate}
			\item $U=\{(x,y):x \in [0,1], y \in [0,2]\}$ und $f:U\to\mathbb{R}$ mit $f(x,y)=x^2+y^2$

				$a=0,b=1,y_u(x)=0,y_o(x)=2$

				\[\int^1_0 \left(\int^{2}_{0} (x^2+y^2)dy\right)dx\]
				\[\int^1_0 \left(\left[x^2y+\frac{y^3}{3}\right]^{2}_{0}\right)dx\]
				\[\int^1_0 \left(2x^2+\frac{2^3}{3}\right)dx\]
				\[\left[\frac{2}{3}x^3+\frac{8}{3}x\right]^1_0\]
				\[\frac{2}{3}+\frac{8}{3}=\frac{10}{3}\]

		\item $U=\{(x,y):x \geq 0, x \leq 1, y \leq x, y \geq 0\}$

			$f:U\to\mathbb{R}$ mit $f(x,y)=x\cdot\sin(y)$

			$a=0,b=1,y_u(x)=0,y_o(x)=x$

			\[\int^1_0 \left(\int^{x}_{0} x\cdot\sin(y)dy\right)dx\]
			\[\int^1_0 \left(\left[x\cdot-\cos(y)\right]^{x}_{0}\right)dx\]
			\[\int^1_0 \left(2x^2+\frac{2^3}{3}\right)dx\]
			\[\left[\frac{2}{3}x^3+\frac{8}{3}x\right]^1_0\]
			\[\frac{2}{3}+\frac{8}{3}=\frac{10}{3}\]
		\end{enumerate}
	\end{beispiel}

	\begin{aufg}
		$f:U\to\mathbb{R}$ mit $f(x,y)=x^2+y$

		\begin{enumerate}
			\item $U=\{(x,y):1 \leq x \leq 2, 2 \leq y \leq 3\}$
			\item $U=\{(x,y):0 \leq x \leq 1, x^2 \leq y \leq \sqrt{x}\}$
		\end{enumerate}
	\end{aufg}
\end{document}          
